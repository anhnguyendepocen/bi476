\documentclass[11pt,a4paper]{article}
\usepackage{CJK}
\usepackage{graphicx}
\usepackage{amsmath,amssymb,amsopn}
\usepackage[english]{babel}
\usepackage[latin1]{inputenc}
\usepackage{url}
\usepackage{enumerate}
\usepackage{color}
\usepackage{times}
\usepackage{xcolor}
\usepackage{geometry}
\usepackage{listings}
\lstset{
language=R,
keywordstyle=\color{blue!70}\bfseries,
basicstyle=\ttfamily,
commentstyle=\ttfamily,
showspaces=false,
showtabs=false,
frame=shadowbox,
rulesepcolor=\color{red!20!green!20!blue!20},
breaklines=true
}

\geometry{paperwidth=18.4cm,paperheight=26cm}
\setlength{\parindent}{0em}
\setlength{\parskip}{1.2\baselineskip}


\title{\small{BI476: Biostatistics - Case Studies}\\
\Large{Assignment 2: Observational Studies}
}
\author{}
\date{Spring, 2018}

\begin{document}
\begin{CJK*}{UTF8}{gbsn}
\maketitle


\section{Case Report and Case Series}
\subsection{True or False}
\begin{enumerate}[(1)]
	\item Case reports are not considered evidence-based study because 
		they involve only one or several patients and are thus not 
		systematic research.
	\item Case reports can demonstrate causality or argue for the adoption 
		of a new treatment approach.
	\item The patient should be described in detail, allowing others to 
		identify patients with similar characteristics.
\end{enumerate}


\section{Cross-sectional Studies}
One hundred healthy troops go on a 1-year mission to a malariandemic area of North Africa. 
During their stay 5 new cases of malaria are dentified, for a rate of 5\%. This number is a:
\begin{enumerate}[(A)]
	\item prevalence rate
	\item mortality rate
	\item incidence-density rate
	\item cumulative-incidence rate
\end{enumerate}

Among untreated children under age 3, 20\% of malaria cases are fatal. 20\% is a:
\begin{enumerate}[(A)]
  \item case-fatality rate
  \item mortality rate
  \item attack rate
  \item prevalence
\end{enumerate}


During the past year, 7 new cases of multiple sclerosis were diagnosed in your 
community of 100,000 people. At any one time during the year, the prevalence of 
multiple sclerosis in your community was probably.
\begin{enumerate}[(A)]
  \item Substantially higher than 7/100,000
  \item Substantially lower than 7/100,000
  \item About 7/100,000
  \item Equal to the cause-specific mortality rate
\end{enumerate}

\section{Case-Control Studies}
\subsection{Read the three articles}
\begin{itemize}
	\item Chambers C.,et al. Selective serotonin-reuptake inhibitors and risk of persistent 
pulmonary hypertension of the newborn. NEJM 2006; 354(6): 579-587.
	\item Smedby K.,et al. Autoimmune and chronic inflammatory disorders and risk of 
non-hodgkin lymphoma by subtype. JNCI 2006;98(1): 51-60.
	\item Teo, K. et al. Tobacco use and risk of myocardial infarction in 52 countries 
in the INTERHEART study: A case-control study. Lancet 2006;368(9536): 647-658.
\end{itemize}
and then answer the following questions.
\begin{enumerate}[(1)]
	\item Is this article paired or not?
	\item Print out the contingency tables.
	\item Can you compute each effect size (OR) of the association, also the corresponding 
		95\% confidence interval, and the $p$-value.
	\item Did the authors use appropriate inclusion and exclusion criteria to avoid effects 
		of confoundings? Why?
\end{enumerate}

\subsection{Answer the question}
How to correct for the effect of confounding factors on the association between the 
outcome and exposure of interest? Can you post some of the commonly-used methods.


\subsection{Choices}
1. Three years ago there was a multistate outbreak of illnesses caused by a specific and 
unusual strain of Listeria monocytogenes. As part of the investigation of this outbreak, 
CDC workers checked the food histories of 20 patients infected with the outbreak strain 
and compared them with the food histories of 20 patients infected with other Listeria 
strains. This study design is best described as which one of the following:
\begin{enumerate}[(A)]
  \item Analytical, experimental
  \item Analytical, observational, case-control
  \item Analytical, observational, cohort
  \item Descriptive
\end{enumerate}

2. The initial studies establishing maternal diethylstilbesterol (DES) intake as a 
cause of vaginal adenocarcinoma in female offspring were case-control studies. 
This was probably largely because:
\begin{enumerate}[(A)]
  \item A couple of decades ago cohort studies hadn't been invented.
  \item A woman taking DES was always rare.
  \item The disease outcome is rare.
  \item The investigators had probably just happened to have a number of cases in their practices.
\end{enumerate}


3. In a case-control study of alcohol intake and bladder cancer, cases and 
matched controls are each interviewed by interviewers who are not blinded 
as to whether the subject is a case or a control. Many of the interviewers 
are in fact convinced that drinking alcohol is a cause of bladder cancer. 
Is this likely to represent a bias?
\begin{enumerate}[(A)]
  \item No, because the interviewers can't affect whether the subjects are considered 
	cases or controls; that's already decided
  \item Yes, but it's hard to predict the direction of the bias.
  \item Yes, and would predispose to a rejection of the null hypothesis.
  \item Yes, and would predispose to an acceptance of the null hypothesis.
\end{enumerate}

\section{Cohort Studies}

\subsection{Choices}

1. A published study follows a large group of women with untreated dysplasia of 
the uterine cervix, documenting the number who improve, stay unchanged, or progress 
into cervical cancer. This study design is best described as which one of the 
following:
\begin{enumerate}[(A)]
  \item Analytic, experimental
  \item Analytic, observational, cohort
  \item Analytic, observational, case/control
  \item Descriptive, observational
\end{enumerate}


2. A community assesses a random sample of its residents by telephone questionnaire. 
Obesity is strongly associated with diagnosed diabetes. This study design is best 
described as which one of the following:
\begin{enumerate}[(A)]
  \item Case-control
  \item Cohort
  \item Cross-sectional
  \item Experimental
\end{enumerate}


\subsection{Cytomegalovirus and coronary restenosis}
Each year cardiologists perform procedures to blocked coronary arteries only to 
have may of these repaired arteries re-clog (restenosis) afterwards. A study 
sponsored by the NIH Heart, Lung and Blood Institute was performed to determine 
whether prior infection with cytomegalovirus was predictive of arterial restenosis. 
In 21 of the 49 patients with serologic evidence of cytomegalovirus infection, 
re-growth of arterial plaque was noted. In contrast, only 2 of the 26 patients 
seronegative patient had restenosis. 
\begin{enumerate}[(1)]
	\item Calculate the risk ratio of restenosis associated with CMV infection. 
		Include a 95\% confidence interval.
	\item Try to interpret results.
	\item Conduct a chi-square test of $H_0: RR = 1$.
\end{enumerate}

\end{CJK*}
\end{document}
