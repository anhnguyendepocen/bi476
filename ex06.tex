% Exam Template for SJTU Department of Bioinformatics and Biostatistics Courses
%
% Using Philip Hirschhorn's exam.cls: http://www-math.mit.edu/~psh/#ExamCls
%
% run pdflatex on a finished exam at least three times to do the grading table on front page.
%
%%%%%%%%%%%%%%%%%%%%%%%%%%%%%%%%%%%%%%%%%%%%%%%%%%%%%%%%%%%%%%%%%%%%%%%%%%%%%%%%%%%%%%%%%%%%%%

% These lines can probably stay unchanged, although you can remove the last
% two packages if you're not making pictures with tikz.
%\documentclass[11pt,addpoints]{exam}
\documentclass[11pt,addpoints]{exam}
\RequirePackage{amssymb, amsfonts, amsmath, latexsym, verbatim, xspace, setspace}
\RequirePackage{tikz, pgflibraryplotmarks, enumerate}
\usepackage{color}
\usepackage{xcolor}
\usepackage{multirow}
\usepackage{CJK}
\usepackage{multicol}
\usepackage{cite}
\usepackage{listings}
\lstset{
language=R,
keywordstyle=\color{blue!70}\bfseries,
basicstyle=\ttfamily,
commentstyle=\ttfamily,
showspaces=false,
showtabs=false,
frame=shadowbox,
rulesepcolor=\color{red!20!green!20!blue!20},
breaklines=true}

% By default LaTeX uses large margins.  This doesn't work well on exams; problems
% end up in the "middle" of the page, reducing the amount of space for students
% to work on them.
\usepackage[margin=1in]{geometry}


\newtheorem{theorem}{Theorem}
\newtheorem{example}{Example}
\newtheorem{definition}{Definition}
\newtheorem{property}{Property}

% Here's where you edit the Class, Exam, Date, etc.
\newcommand{\class}{BI476 Biostatistics - Case Study}
\newcommand{\term}{Spring 2018}
\newcommand{\examnum}{Exercise 6}
\newcommand{\examdate}{04/20/2018}
\newcommand{\timelimit}{60 Minutes}

% For an exam, single spacing is most appropriate
\singlespacing
% \onehalfspacing
% \doublespacing

% For an exam, we generally want to turn off paragraph indentation
\parindent 0ex

\newcommand{\tf}[1][{}]{%
\fillin[#1][0.25in]%
}

\begin{document} 
\begin{CJK*}{UTF8}{gbsn}

% These commands set up the running header on the top of the exam pages
\pagestyle{head}
\firstpageheader{}{}{}
\runningheader{\class}{\examnum\ - Page \thepage\ of \numpages}{\examdate}
\runningheadrule

\begin{center}
{\LARGE Exercise 6: Linear Models and Generalizations}\\
{\Large 2018 Spring}
\end{center}

\rule[1ex]{\textwidth}{.1pt}

\section{Background Knowledge}

\subsection{Maximum Likelihood Estimator}


\subsection{Exponential Families}
A one-parameter exponential-family distribution has the probability density function (pdf) of the following form:
$$
f(y;\theta) = \exp \left( \frac{y\theta - b(\theta)}{a(\phi)} + c(\phi,y) \right)
$$
where
\begin{itemize}
	\item $\theta$ is the \textbf{canonical parameter}.
	\item $\phi$ is the (optional) \textbf{dispersion parameter}.
	\item The expected value of $Y$: $E(Y) = \mu = b'(\theta)$
	\item The variance of $\mu$ is: $V(\mu) = b''(\theta)$
	\item The variance of $Y$ is: $Var(Y) = V(\mu) a(\phi)$
	\item The \textbf{link function} $g(\mu) = \eta = x^T \beta$
	\item \textbf{Canonical link function} is obtained through $\eta = \theta$. 
\end{itemize}

Thus the log-likelihood function of the data is:
$$
l(\theta;y) = \frac{y\theta - b(\theta)}{a(\phi)}
$$


\subsection{Generalized Linear Models (GLMs) Fitting}

With this we can obtain the \textbf{Fisher's score vector} $\mathbf{u} = (u_j)$ with
$$
u_j = \frac{\partial l}{\partial \beta_j} = \frac{\partial l}{\partial \theta} \frac{\partial \theta}{\partial \mu} \frac{\partial \mu}{\partial \eta} \frac{\partial \eta}{\partial \beta_j}
$$

Since
$$
\begin{array}{lcl}
\frac{\partial l}{\partial \theta} &=& \frac{y - b'(\theta)}{a(\phi)}\\
\frac{\partial \theta}{\partial \mu} &=& \frac{1}{b''(\theta)} = \frac{1}{V(\mu)} = \frac{a(\phi)}{Var(y)}\\
\frac{\partial \eta}{\partial \beta_j} &=& x_{ij}
\end{array}
$$
Therefore
$$
\frac{\partial l}{\partial \beta_j} = \frac{y-\mu}{Var(y)} \left( \frac{\partial \mu}{\partial \eta} \right)x_{ij}
$$

When we use the canonical link function
$$
\frac{\partial \mu}{\partial \eta} = \frac{\partial \mu}{\partial \theta} = b''(\theta)
$$
therefore
$$
\frac{\partial l}{\partial \beta_j} = \frac{y-\mu}{Var(y)}b''(\theta)x_{ij} = \frac{y-\mu}{a(\phi)} x_{ij}
$$


And \textbf{Fisher's information matrix} can be obtained by:
$$
\begin{array}{lcl}
-E\left( \frac{\partial^2 l}{\partial \beta_j \partial \beta_k} \right) &=& E\left[ \frac{\partial l}{\partial \beta_j} \frac{\partial l}{\partial \beta_k}\right]\\
&=& E\left( \frac{y-\mu}{Var(y)}\right)^2 \left( \frac{\partial \mu}{\partial \eta}\right)^2 x_{ij} x_{ik}\\
&=& \frac{1}{Var(y)} \left( \frac{\partial \mu}{\partial \eta} \right)^2 x_{ij} x_{ik}
\end{array}
$$

For general link function, the score function for only \textbf{1-observation} becomes
$$
\frac{\partial l}{\partial \beta_j} = \frac{y-\mu}{Var{y}} \left( \frac{\partial \mu}{\partial \eta}\right) x_{ij}
$$

And the score function for $n$-observation becomes:
$$
\frac{\partial l}{\partial \beta} = X^T A(y-\mu)
$$

Similarly the Fisher's information matrix can also be simplified as:
$$
-E\left( \frac{\partial^2 l}{\partial \beta_j \beta_k} \right) = \frac{1}{Var(y)} \left( \frac{\partial \mu}{\partial \eta}\right)^2 x_{ij} x_{ik}
$$
and also the matrix form:
$$
-E\left( \frac{\partial^2 l}{\partial \beta \partial \beta^T} \right) = X^T W X
$$
where
$$
W = \text{diag}(w_1, \dots, w_n)
$$
and 
$$
w_i = \frac{1}{Var(y_i)} \left( \frac{\partial \mu_i}{\partial \eta_i}\right)^2 = [b''(\theta_i)]^{-1} \left( \frac{\partial \eta_i}{\partial \mu_i}\right)^{-2}
$$



\subsection{Iteratively Reweighted Least Squares (IRWLS)}
Fisher's scoring algorithm can iteratively compute the coefficient by:
$$
\beta^{(t+1)} = \beta^{(t)} + (X^T W X)^{-1} X^T A (y - \mu) 
$$

The score equation can be solved using the numerical method (Newton-Raphson), iteratively reweighted least squares (IRWLS):
$$
\beta^{(t+1)} = \left( X^T W X \right)^{-1} \left[ X^T W X \beta^{(t)} + X^T A (y - \mu)\right]
$$

Since $X\beta = \eta$, we have
$$
A = W\left( \frac{\partial \eta}{\partial \mu}\right)
$$

Replace it into the equation:
$$
\beta^{(t+1)}  = \left( X^T W X \right)^{-1} X^T W z
$$
which is similar to the closed-form of least squares fitting, where
$$
z = \eta + \left( \frac{\partial \eta}{\partial \mu}\right) (y-\mu)
$$
is called the \textbf{adjusted dependent variable}.


\begin{example}[Logistic Regression]

Let $y_1, \dots, y_n$ where $y_i \sim \text{Bin}(n_i, p_i)$
$$
\begin{array}{lcl}
f(y; p) = \binom{n}{y} p^y (1-p)^{n-y} &=& \exp \left( y\log p + (n-y)\log(1-p) + \log \binom{n}{y} \right) \\
&=& \exp\left(y\log \frac{p}{1-p} + n\log(1-p) + \log \binom{n}{y} \right)
\end{array}
$$
\begin{itemize}
	\item $\theta = \log \frac{p}{1-p} \Rightarrow p = \frac{e^{\theta}}{1+e^{\theta}}$;
	\item $b(\theta) = n \log \left(\frac{1}{1+e^{\theta}}\right)$
	\item $\mu = b'(\theta) = n \frac{e^{\theta}}{1+e^{\theta}} = np$
	\item $Var(\mu) = np(1-p)$
\end{itemize}
For the canonical link function:
$$
\eta = \theta = \log \frac{p}{1-p} = \log \frac{\mu}{n-\mu}
$$
\end{example}

\section{Exercises}

\begin{questions}

\question[20]
For a set of independent observations $(Y_1, \dots, Y_N), N=2n$; $Y_i \sim Bin(n_i, p_i), 
i=1,\dots,N$ ($n_i$ known); We can fit a GLM:
$$
\log \frac{p_i}{1-p_i} = x_i^T \beta
$$
where the design matrix has the form
$$
X = \begin{bmatrix}
a_1 & 0\\
\vdots & \vdots\\
a_n & 0\\
0 & c_1\\
\vdots & \vdots\\
0 & c_n
\end{bmatrix}
$$
and the parameter $\beta = (\beta_1, \beta_2)^T$.

\begin{parts}
\part[10]
Find the score vector $u$ in terms of $Y_i, n_i, p_i, a_i, c_i$ and show 
that $\hat{\beta}_1$ are independent of $Y_{n+1}, \dots, Y_N$ while 
$\hat{\beta}_2$ are independent of $Y_1,\dots,Y_n$.
\begin{solution}
The log-likelihood function is
$$
\begin{array}{lcl}
\ell(\beta;Y) &=& \sum_{i=1}^N \left( Y_i \log\frac{p_i}{1-p_i} + n_i \log(1-p_i) + \log \binom{n_i}{Y_i}\right)\\
&=& \sum_{i=1}^N \left( Y_i x_i^T \beta - n_i \log(1+e^{x_i^T \beta}) + \log \binom{n_i}{Y_i}\right)
\end{array}
$$
and the score vector:
$$
\begin{array}{lcl}
u_1 &=& \partial \ell /\partial \beta_1\\
&=& \sum_{i=1}^n (Y_i a_i - n_i \frac{e^{x_i^T\beta}}{1+e^{x_i^T\beta}}a_i)\\
&=& \sum_{i=1}^n (Y_i - n_i p_i)a_i\\
u_2 &=& \partial \ell / \partial \beta_2\\
&=& \sum_{i=n+1}^N (Y_i c_{i-n} - n_i \frac{e^{x_i^T\beta}}{1+e^{x_i^T\beta}}c_{i-n})\\
&=& \sum_{i=n+1}^N (Y_i - n_i p_i)c_{i-n}
\end{array}
$$
thus we obtain
$$
u = (u_1, u_2)^T = \left( \sum_{i=1}^n (Y_i - n_i p_i)a_i, \sum_{i=n+1}^N (Y_i - n_i p_i)c_{i-n} \right)^T
$$
Since $p_1, \dots, p_n$ depend only on $\beta_1$ and $p_{n+1}, \dots, p_N$ depend only on 
$\beta_2$, so do $u_1$ and $u_2$ respectively.

Then we can obtain the maximum likelihood estimator $\hat{\beta}_1$ and $\hat{\beta}_2$ by 
solving $u_1(\beta_1)=0$ and $u_2(\beta_2)$ separately.
\end{solution}


\part[10]
Find the Fisher's information matrix and show that the method of scoring for $\hat{\beta}$ 
has the form:
$$
\begin{array}{lcl}
\hat{\beta_1}^{(k)} = \hat{\beta}_1^{(k-1)} + \frac{\sum_{i=1}^n (Y_i - n_ip_i^{(k-1)})a_i}{\sum_{i=1}^n n_ip_i^{(k-1)}(1-p_i^{(k-1)})a_i^2}\\
\hat{\beta_2}^{(k)} = \hat{\beta}_2^{(k-1)} + \frac{\sum_{i=n+1}^N (Y_i - n_ip_i^{(k-1)})c_{i-n}}{\sum_{i=n+1}^N n_ip_i^{(k-1)}(1-p_i^{(k-1)})c_{i-n}^2}
\end{array}
$$
\begin{solution}
The Fisher's information matrix is
$$
I = \begin{bmatrix}
\partial^2 \ell / \partial \beta_1^2 & \partial^2 \ell / \partial \beta_1 \partial \beta_2\\
\partial^2 \ell / \partial \beta_1 \partial \beta_2 & \partial^2 \ell / \partial \beta_2^2 
\end{bmatrix} = \begin{bmatrix}
\sum_{i=1}^n n_i p_i(1-p_i)a_i^2 & 0\\
0 & \sum_{i=n+1}^N n_p p_i(1-p_i)c_{i-n}^2
\end{bmatrix}
$$
Thus we can get
$$
I^{-1} u = \left[ \frac{\sum_{i=1}^n (Y_i - n_ip_i)a_i}{\sum_{i=1}^n n_i p_i(1-p_i)a_i^2}, \frac{\sum_{i=n+1}^N (Y_i - n_ip_i)c_{i-n}}{\sum_{i=n+1}^N n_i p_i(1-p_i)c_{i-n}^2}\right]^T
$$
and we can easily get the above result.
\end{solution}
\end{parts}


\question[30]
We have n independent observations $Y_1, \dots, Y_n$ following normal distributions with 
the same variance $\sigma^2$:
$$
\begin{aligned}
EY_1 = \mu_1 = \beta_1 + \beta_2\\
EY_2 = EY_3 = \dots = EY_n = \mu = \beta_1
\end{aligned}
$$
where $\beta = (\beta_1, \beta_2)^T$ are parameters of interest.
\begin{parts}

\part[10]
What is the design matrix $X$ in this model?
\begin{solution}
The design matrix is a $n \times 2$ matrix in this form:
$$
X = \begin{bmatrix}
1 & 1\\
1 & 0\\
\vdots & \vdots\\
1 & 0
\end{bmatrix}
$$
\end{solution}


\part[10]
Show that only the first observation is highly influential, using the simple rule that 
$h_{ii} > 2p/n$ where $H=[h_{ij}]$ is the hat matrix $H=X(X^TX)^{-1}X^T$, and $p$ is the 
number of paramters, here we have $p=2$.
\begin{solution}
In this example, we have
$$
X^T X = \begin{bmatrix}
n & 1\\
1 & 1
\end{bmatrix}, (X^T X)^{-1} = \frac{1}{n-1} \begin{bmatrix}
1 & -1\\
-1 & n
\end{bmatrix}
$$
and therefore the hat matrix can be computed:
$$
H = X(X^T X)^{-1} X^T = \begin{bmatrix}
1 & 0 & \dots & 0\\
0 & 1/(n-1) & \dots & 1/(n-1)\\
\vdots & \vdots & \ddots & \vdots\\
0 & 1/(n-1) & \dots & 1/(n-1)
\end{bmatrix}
$$
Thus we have:
$$
h_{11} = 1 > 2p/n
$$
while
$$
h_ii = 1/(n-1) < 2p/n, i=2,\dots,n
$$
\end{solution}

\part[10]
Find the maximum likelihood estimator of $\beta$.
\begin{solution}
In this situation the MLE estimator has the closed-form:
$$
\hat{\beta} = (X^T X)^{-1}X^T Y
$$
we can easily get
$$
(X^TX)^{-1}X^T = \frac{1}{n-1} \begin{bmatrix}
0 & 1 & 1 & \dots & 1\\
n-1 & -1 & -1 & \dots & -1
\end{bmatrix}
$$
Thus
$$
\hat{\beta} = (X^TX)^{-1}X^T Y = \begin{bmatrix}
\frac{1}{n-1} \sum_{i=2}^n Y_i\\
Y_1 - \frac{1}{n-1}\sum_{i=2}^n Y_i
\end{bmatrix}.
$$
\end{solution}


\end{parts}


\question[20]
Table 3 from \cite{NEJM2007case} displays results from a case-control study of oropharyngeal cancer patients. 
The investigators were looking for associations between HPV and oropharyngeal cancer. Use the table to answer 
the following questions.

\begin{parts}
\part[5]
Draw a conclusion on the association between the seropositive HPV-16 L1 serologic status and oropharyngeal cancer.
\begin{solution}
There is a 32.2-fold adjusted increase in the odds of oropharyngeal cancer for those with Seropositive HPV-16 L1 
serologic status and it is statistically significant.
\end{solution}

\part[5]
Can you calculate the unadjusted risk ratio for the risk of oropharyngeal cancer in patients who were positive 
for oral HPV-16 infection, using only the information given in this table?
\begin{solution}
Cannot calculate from the information given in this table.
\end{solution}

\part[5]
What statistical method was used to calculate the “adjusted odds ratios” given in the table? The unadjusted odds 
ratio for HPV-16 L1 seropositivity is 17.6 but the adjusted odds ratio is 32.2.  How do you explain this difference?
\begin{solution}
Logistic regression. The unadjusted odds ratio was an underestimate—which could happen if some of the confounders 
were inversely related to exposure or disease.
\end{solution}

\part[5]
Which statistical test was used to compare the oral HPV infection prevalence in cases versus controls? Write down the 
correct statistic for comparing.
\begin{solution}

\end{solution}

\end{parts}

\question[20]
Table 2 from \cite{JNNP2009association} displays the beta coefficients from linear regression analysis. Refer to this table 
to answer the following questions.

\begin{parts}

\part[5]
What is your interpretation of the Beta coefficient relating BMI ($kg/m^2$) to vitamin D levels 
[Beta coefficient (and 95\% CI) = -0.811 (-1.081, -0.541)]?
\begin{solution}
Every 1 $kg/m^2$ increase in BMI is associated with an average 0.811 nmol/L decrease in vitamin D 
levels, after adjusting for age.
\end{solution}

\part[5]
Specify the models used to generate the Beta coefficient relating BMI ($kg/m^2$) to vitamin D levels?
\begin{solution}
$$
\text{Vitamin D} = \beta_0 + \beta_{\text{age}}\times \text{age} + \beta_{\text{BMI}} \times \text{BMI}
$$
\end{solution}

\part[5]
Which factor is associated with the greatest decrease in DSST score?
\begin{solution}
A 10-year increase in age.
\end{solution}

\part[5]
What is another test that the authors could use to determine whether people in the different BDI 
categories (normal, mild to borderline, moderate to extreme) have significantly different mean DSST 
scores?
\begin{solution}
Analysis of variance (ANOVA).
\end{solution}
\end{parts}


\question[10]
A multiple regression model is fitted on a data set:
$$
Y_i = x_i^T \beta + \epsilon_i, \epsilon_i \stackrel{\text{i.i.d}}{\sim} \mathbf{N}(0,\sigma^2)
$$
where $x_i = (1,x_{i1}, \dots, x_{i4})^T, \beta=(\beta_0, \beta_1, \dots, \beta_4)^T$ 
and we can obtain the result through R:
\begin{lstlisting}

Coefficients:
              Estimate   Std. Error  t value  Pr(>|t|)
(Intercept)        ???       0.1960    8.438  3.57e-13
x1              5.3036       2.5316      ???  0.038834
x2              4.0336       2.4796    1.627  0.107111
x3             -9.3153       2.4657   -3.778  0.000276
x4              0.5884       2.2852    0.257  0.797373

Residual standard error: 1.892 on 95 degrees of freedom
Multiple R-squared: 0.1948, Adjusted R-squared: ???
F-statistic: 5.745 on 4 and 95 DF,  p-value: 0.0003483
\end{lstlisting}

\begin{parts}

\part[2]
What is the value of the $t$-statistics of $\hat{\beta}_1$?
\begin{solution}
2.095
\end{solution}

\part[2]
How many observations are in the data set?
\begin{solution}
100
\end{solution}

\part[2]
Has the null hypothesis $H_0: \beta_3 = 0$ to be rejected on $\alpha=0.05$?
\begin{solution}
Yes
\end{solution}

\part[2]
What is the estimate of the intercept $\hat{\beta}_0$?
\begin{solution}
1.654
\end{solution}

\part[2]
What is the estimate of $Var(\epsilon_i)$?
\begin{solution}
3.579
\end{solution}

\part[2]
What is the 95\% confidence interval for $\beta_3$?
\begin{solution}
$$
-9.315 \pm 1.99 \times 2.466
$$
\end{solution}
\end{parts}


\end{questions}

% bibliographystyle{}
% plain,按字母的顺序排列,比较次序为作者、年度和标题.
% unsrt,样式同plain,只是按照引用的先后排序.
% alpha,用作者名首字母+年份后两位作标号,以字母顺序排序.
% abbrv,类似plain,将月份全拼改为缩写,更显紧凑.
% ieeetr,国际电气电子工程师协会期刊样式.
% acm,美国计算机学会期刊样式.
% siam,美国工业和应用数学学会期刊样式.
% apalike,美国心理学学会期刊样式.

\bibliographystyle{plain}
\bibliography{quiz6}

\end{CJK*}
\end{document}
