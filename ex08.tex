% Exam Template for SJTU Department of Bioinformatics and Biostatistics Courses
%
% Using Philip Hirschhorn's exam.cls: http://www-math.mit.edu/~psh/#ExamCls
%
% run pdflatex on a finished exam at least three times to do the grading table on front page.
%
%%%%%%%%%%%%%%%%%%%%%%%%%%%%%%%%%%%%%%%%%%%%%%%%%%%%%%%%%%%%%%%%%%%%%%%%%%%%%%%%%%%%%%%%%%%%%%

% These lines can probably stay unchanged, although you can remove the last
% two packages if you're not making pictures with tikz.
%\documentclass[11pt,addpoints]{exam}
\documentclass[11pt,addpoints,answers]{exam}
\RequirePackage{amssymb, amsfonts, amsmath, latexsym, verbatim, xspace, setspace}
\RequirePackage{tikz, pgflibraryplotmarks, enumerate}
\usepackage{color}
\usepackage{xcolor}
\usepackage{multirow}
\usepackage{CJK}
\usepackage[T1]{fontenc}
\usepackage{multicol}
\usepackage{cite}
\usepackage{longtable}
\usepackage{listings}
\lstset{
language=R,
keywordstyle=\color{blue!70}\bfseries,
basicstyle=\ttfamily,
commentstyle=\ttfamily,
showspaces=false,
showstringspaces=false,
showtabs=false,
frame=shadowbox,
rulesepcolor=\color{red!20!green!20!blue!20},
breaklines=true}

% By default LaTeX uses large margins.  This doesn't work well on exams; problems
% end up in the "middle" of the page, reducing the amount of space for students
% to work on them.
\usepackage[margin=1in]{geometry}


\newtheorem{theorem}{Theorem}
\newtheorem{example}{Example}
\newtheorem{definition}{Definition}
\newtheorem{property}{Property}

% Here's where you edit the Class, Exam, Date, etc.
\newcommand{\class}{BI476 Biostatistics - Case Study}
\newcommand{\term}{Spring 2018}
\newcommand{\examnum}{Exercise 8}
\newcommand{\examdate}{06/15/2018}
\newcommand{\timelimit}{60 Minutes}

% For an exam, single spacing is most appropriate
\singlespacing
% \onehalfspacing
% \doublespacing

% For an exam, we generally want to turn off paragraph indentation
\parindent 0ex

\newcommand{\tf}[1][{}]{%
\fillin[#1][0.25in]%
}

\begin{document} 
\begin{CJK*}{UTF8}{gbsn}

% These commands set up the running header on the top of the exam pages
\pagestyle{head}
\firstpageheader{}{}{}
\runningheader{\class}{\examnum\ - Page \thepage\ of \numpages}{\examdate}
\runningheadrule

\begin{center}
{\LARGE Exercise 8: Semiparametric Survival Analysis}\\
{\Large 2018 Spring}
\end{center}

\rule[1ex]{\textwidth}{.1pt}

\section{Cox proportional hazards model}
A Cox proportional hazards model specifies the relationship between the hazards and 
the covariates:
$$
h(t,\mathbf{X}) = h_0(t) \exp\left( \sum_{i=1}^p x_i \beta_i \right)
$$
\begin{itemize}
	\item $h_0(t)$: \textbf{baseline hazard rate}, time-dependent.
	\item $\mathbf{X}$: vector of explanatory variables, covariates.
	\item $\exp(\beta_i)$: \textbf{hazard ratio} for the coefficient $\beta_i$.
	\item The \textbf{ratio} between the predicted \textbf{hazard rate} of 
		two individuals that differ by 1 unit of $x_i$. 
	\item $\mathbf{X}\beta$: \textbf{Prognostic index (预后指数)} 
\end{itemize}
There is a parameter $\beta$, but no parametric form for $h_0(t)$ in the model. That's 
the reason why we call it \textbf{semi-parametric model}.

Thus:
$$
\begin{array}{lcl}
\log h(t, \mathbf{X}) &=& \log h_0(t) + \mathbf{X}\beta\\
\log \frac{h(t,X)}{h_0(t)} &=& \mathbf{X}\beta
\end{array}
$$

\section{Accerlated Failure Time (AFT) Model}

\section{Cox's Proportional Hazards Model}

In this section we will introduce the Cox's proportional hazards model, give a heuristic 
development of the \textbf{partial likelihood function}, and discuss adaptions to 
accomodate \textbf{tied observations}. We then explore some specific tests that arise 
from likelihood-based inferences based on the partial likelihood. Asymptotic properties 
of the resulting estimators and tests will also be covered.

\subsection{PH Model}
For $n$ subjects with covariate vector $Z$ and the survival outcome $(U, \delta)$ 
representing noninformatively right-censored values of a survival time $T$. That is, 
for subject $i$,
\begin{itemize}
	\item $Z_i$: the covariate vector;
	\item $T_i$: the underlying survival time;
	\item $C_i$: the potential censoring time;
	\item $U_i = \min(T_i, C_i)$;
	\item $\delta_i = I(T_i \le C_i)$;
	\item $T_i \perp C_i | Z_i$
\end{itemize}

One way to model a relationship between $Z$ and $T$ is by assuming \textbf{$h(\cdot)$ 
is functionally related to $Z$}:
$$
T \sim \exp(\lambda_Z)
$$
where $h(t) = \lambda_Z = \exp(\alpha + \beta Z) = \lambda_0 \exp(\beta Z)$ ($\lambda_0 = \exp(\alpha)$).

Thus, we might assume that $T_i \sim \exp(\lambda_0 \exp(\beta Z_i))$. Therefore, 
$\beta=0$ means that $\lambda_Z$ does not depend on $Z$, and also $Z$ is not assocated 
with $T$.

\subsubsection{Generalization}
Let $h(t|Z)$ denote the hazard function for a subject with covariate $Z$. Suppose that
$$
h(t | Z) = h_0(t) \times g(Z)
$$
where $h_0(t)$ is a function of $t$ without $Z$, and $g(Z)$ is a function of $Z$ without 
$t$.

This can be called \textbf{multiplicative hazards model} or \textbf{proportional hazards 
model}. This factorization implies that
$$
\frac{h(t|Z=Z_1)}{h(t|Z=Z_2)} = \frac{g(Z_1)}{g(Z_2)}
$$
which is independent of $t$. That is the reason why it is called the \textbf{proportional 
hazards (PH) model}.

\begin{example}{({\bf Cox's proportional hazards (Cox's PH) model})}
$$
g(Z) = \exp(\beta Z) \Rightarrow h(t|Z) = h_0(t) \times \exp(\beta Z)
$$
where
$$
\frac{h(t|Z=Z_1)}{h(t|Z=Z_2)} = \exp(\beta(Z_1 - Z_2))
$$
For a scalar $Z$, $\exp(\beta) = \mbox{hazard ratio corresponding to a unit change in }Z$.
\end{example}

\begin{example}{({\bf Categorical $Z$})}
$$
Z = \begin{pmatrix}
Z_1\\
Z_2
\end{pmatrix}, Z_1 = \begin{cases}
0 & R_x (\mbox{treatment}) = 0\\
1 & R_x (\mbox{treatment}) = 1
\end{cases}, Z_2 = \begin{cases}
0 & \mbox{female}\\
1 & \mbox{male}
\end{cases}
$$
and $\beta = (\beta_1, \beta_2)$, 
then
$$
h(t|Z) = \begin{cases}
h_0(t) & R_x = 0, \mbox{female}\\
h_0(t)\exp(\beta_1) & R_x = 1, \mbox{female}\\
h_0(t)\exp(\beta_2) & R_x = 0, \mbox{male}\\
h_0(t)\exp(\beta_1+\beta_2) & R_x=1, \mbox{male}
\end{cases}
$$
\end{example}

\subsection{Inference}

Our inferential problems include:
\begin{itemize}
	\item Estimate $\beta$ and derives its statistical properties,
	\item Testing hypothesis $H_0: \beta=0$ or for part of $\beta$,
	\item Diagnostics of the assumptions
\end{itemize}


\subsubsection{Estimation}
How can we infer the coefficients $\beta$?
\begin{itemize}
	\item Assumption 1: parametric form of $h_0(t)$, conduct parametric analysis 
		(e.g., $h_0(t) = \lambda_0$)
	\item Assumption 2: arbitrary $h_0(t)$
\end{itemize}

The survival function:
$$
S(u|Z) = (S_0(u))^{\exp(\beta Z)}
$$
where 
$$
\begin{array}{lcl}
S_0(u) &=& \exp(-\int_0^u h_0(t) dt)\\
&=& \mbox{survival function for someone with }Z=0\\  
&=& S(u|0)
\end{array}
$$
Also, we have $f(u|Z) = h(u|Z) S(u|Z)$.

Therefore , for $n$ independent observations $(u_i, \delta_i, z_i), i=1,\dots, n$, the 
likelihood function is
$$
\begin{array}{lcl}
L(\beta, h_0(\cdot)) &=& \prod_{i=1}^n f(u_i|z_i)^{\delta_i} S(u_i | z_i)^{1-\delta_i}\\
&=& \prod_{i=1}^n h(u_i|z_i)^{\delta_i} S(u_i|z_i)\\
&=& \prod (h_0(u_i)e^{\beta z_i})^{\delta_i} (e^{-\int_0^{u_i} h_0(t)dt})^{e^{\beta z_i}}\\
&=& f(\mbox{data}, \beta, h_0(\cdot))
\end{array}
$$

If we allow arbitrary $h_0(\cdot)$, then the \textbf{parameter space} is
$$
\mathcal{H} \times \mathbb{R}^p = \left\{ (h_0(\cdot), \beta): h_0(u) \ge 0 \mbox{ for all }u, \int_0^{\infty} h_0(u) = \infty \mbox{ and } \beta \in \mathbb{R}^p \right\}
$$
where $p$ is the dimension of the vector $\beta$. The condition
$$
\int_0^{\infty} h_0(u) du = \infty
$$
ensures that $S_0(\infty) = 0$.

In many application, the main goal is to make an inference about $\beta$ and the 
underlying hazard $h_0(\cdot)$ is a nuisance function. Inference in such a settings 
are commonly called \textbf{semi-parametric}. Therefore, the standard likelihood 
theory, based on Euclidean parameter spaces, does not apply here.

\subsubsection{Cox's partial likelihood estimator}

Try to factorize $L(\beta, h_0(\cdot))$ into
$$
L(\beta, h_0(\cdot)) = L_1(\beta) \times L_2(\beta, h_0)
$$
where
\begin{itemize}
	\item $L_1(\beta)$ is a function of $\beta$, whose maximum estimate $\hat{\beta}$ enjoys 
		nice properties $(\hat{\beta} \stackrel{P}{\to} \beta)$ and $(\sqrt{n} 
		(\hat{\beta} - \beta) \stackrel{\mathcal{L}}{\to} N)$ although perhaps 
		inefficient.
	\item $L_2(\beta,h_0(\cdot))$ is a function of $h_0(\cdot)$ and $\beta$ which 
		contains relatively little information about $\beta$.
\end{itemize}

Then, Cox recommends to infer $\beta$ on the partial likelihood function $L_1(\beta)$.


\subsection{Partial Likelihood Function, $L_p$}

$$
\begin{array}{lcl}
L_p = \prod_{i=1}^d q_i &=& \prod_{i=1}^d \frac{h_0(t_i) \exp(\mathbf{X}_i \beta)}{\sum_{j \in R_i} h_0(t_i) \exp(\mathbf{X}_j \beta)}\\
&=& \prod_{i=1}^d \frac{\exp(\mathbf{X}_i \beta)}{\sum_{j \in R_i} \exp(\mathbf{X}_j \beta)}
\end{array}
$$
where
\begin{itemize}
	\item $d$: number of non-censored time points.
	\item $R_i$: risket set at time $t_i$.
	\item $q_i = \frac{h_i(t)}{\sum_{j \in R_i} h_j(t)}$ is the probability of failure at time $t_i$ 
	\item Only the non-censored subjects are included in the analysis for numerator.
	\item In the denominator term, all the subject in risk (including the censored) are 
		included into the computation.
\end{itemize}

This function can also be rewritten as (for computational convenience):
$$
L_p = \prod_{i=1}^n \left( \frac{\exp(\mathbf{X}_i \beta)}{\sum_{j \in R_i} \exp(\mathbf{X}_j \beta)} \right)^{\delta_i}
$$
where
$$
\delta_i = \begin{cases}
1 & \mbox{subject }i \mbox{failed}\\
0 & \mbox{otherwise}
\end{cases}
$$

\subsubsection{Log partial likelihood function}
$$
\begin{array}{lcl}
l(\beta) &=& \log L_p\\
&=& \sum_{i=1}^d \mathbf{X}_i \beta - \sum_{i=1}^d \log \left( \sum_{j \in R_i} \mathbf{X}_j\beta \right)
\end{array}
$$
Let $\frac{\partial l}{\partial \beta} = 0$, we can obtain the regression coefficients $\mathbf{\beta}$ 
through \textbf{Newton-Raphson} iterative approach.


\subsection{Estimate of Coefficients and Hypothesis Tests}
\begin{example}{({\bf Survival analysis of nasal lymphoma patients})}
Here is the follow-up data of 16 nasal lymphoma patients in a hospital:

\begin{tabular}{lcccccccc}
\hline
id & gender & age & stage & bleed & rdx & chmx & days & status\\
\hline
1 & 1 & 45 & 2 & 2 & 0 & 1 & 578 & 1\\
2 & 0 & 36 & 2 & 2 & 0 & 1 & 1549 & 1\\
3 & 1 & 57 & 2 & 2 & 1 & 0 & 938 & 1\\
4 & 0 & 45 & 2 & 0 & 1 & 0 & 4717 & 0\\
5 & 0 & 42 & 2 & 0 & 1 & 1 & 4111 & 1\\
6 & 0 & 39 & 2 & 1 & 0 & 1 & 1245 & 1\\
7 & 1 & 38 & 2 & 1 & 1 & 1 & 4435 & 1\\
8 & 1 & 45 & 2 & 2 & 1 & 0 & 3750 & 1\\
9 & 1 & 30 & 2 & 0 & 1 & 0 & 3958 & 1\\
10 & 0 & 45 & 2 & 1 & 0 & 1 & 2581 & 1\\
11 & 0 & 45 & 3 & 1 & 0 & 1 & 3572 & 1\\
12 & 1 & 57 & 2 & 1 & 1 & 0 & 2938 & 1\\
13 & 0 & 57 & 2 & 2 & 0 & 1 & 1932 & 1\\
14 & 1 & 49 & 2 & 2 & 1 & 1 & 3205 & 1\\
15 & 1 & 33 & 2 & 1 & 0 & 1 & 3451 & 1\\
16 & 0 & 51 & 2 & 2 & 1 & 0 & 2363 & 1\\
\hline
\end{tabular}

Analyze the data using the Cox's proportional hazards model, using the other six metrics as 
covariates:
\begin{lstlisting}
library(survival)
lympho <- read.table("data/nasallym.dat", header=T)
for (i in c(2,4:7)){
	lympho[,i] <- factor(lympho[,i])
}
cox.mod <- coxph(Surv(days, status) ~ gender + age + stage + bleed + rdx + chmx, 
		data = lympho)
summary(cox.mod)
\end{lstlisting}
\end{example}


\subsection{Cox's PH Models with Time-dependent Covariates}

Since survival data is a time-series data, some covariates may also change over time, 
which we refer to as \textbf{time-dependent covariates}.

Here are some examples:
\begin{itemize}
	\item Cumulative exposure to some risk factor,
	\item Smoking status,
	\item Heart (kidney) transplant status,
	\item Blood pressure
\end{itemize}

We might have more than one such covariate. For the $i$-th participants, we denote such 
covariates as:
$$
Z_i(t) = (Z_{i1}(t), \dots, Z_{iq}(t))^T
$$
If the $j$-th covariate is time-independent, then $Z_{ij}(t)$ is constant over time.

Let $Z_i^H(t)$ denote the history of the time-dependent covariates up to time $t$,
$$
Z_i^H(t) = \{Z_i(u), 0 \le u \le t\},
$$
then we can define the hazard rate at time $t$ conditional on this history:
$$
\lambda(t|Z_i^H(t)) = \lim_{\Delta t \to 0} \frac{P[t \le T_i \le t+\Delta t | T_i \ge t, Z_i^H(t)]}{\Delta t}
$$
This is the instantaneous rate of failure at time $t$, given that the individual was at 
risk at time $t$ with a history of $Z_i^H(t)$. For such a conditional hazard rate, we 
may consider a proportional hazards model:
$$
\lambda(t|Z_i^H(t)) = \lambda_0(t) \exp(\beta^T g(Z_i^H(t))),
$$
More often we will choose $g(\cdot)$ as $g(Z_i^H(t)) = z_i(t)$, then
$$
\lambda(t|Z_i^H(t)) = \lambda_0(t) \exp(\beta^T Z_i(t))
$$
if we implicitly assume that the hazard rate at time $t$ given the entire history of 
the covariates up to time $t$ is only affected by the current values of the covariates 
at time $t$.


\begin{example}{({\bf The effect of exposure to asbestos over time on mortality})}
A sample of workers in a factory where asbestos is made were monitored for a period of 
time and data were collected on survival and asbestos exposure.

For the $i$-th individuals, the data could be summarized as
$$
(X_i, \delta_i, Z_i^H(X_i))
$$
where
\begin{itemize}
	\item $X_i = \min(T_i, C_i)$ is the observed survival time or censoring time,
	\item $\delta_i = I(T_i \le C_i)$ is the failure indicator,
	\item $Z_i^H(X_i)$ is the history of asbestos exposure up to time $X_i$
\end{itemize}

\end{example}

Suppose we wish to use the above proportional hazards model with time-dependent covariates,
then what should we use for the function $g(Z_i^H(t))$?
\begin{itemize}
	\item Use the cumulative exposure (need extrapolation), i.e.
	$$
g(Z_i^H(t)) = \sum_j Z_i(u_{ij})(u_{ij} - u_{i(j-1)}),
	$$
	\item Use average exposure up to time $t$
	$$
g(Z_i^H(t)) = \frac{\sum_{u_{ij} < t}Z_i(u_{ij})}{\mbox{\# of measurements up to }t}
	$$
	\item Use maximum exposure up to time $t$
	$$
g(Z_i^H(t)) = \max\{Z_i(u_{ij}): u_{ij} < t\}
	$$
	\item Use the compound exposure term.
\end{itemize}

If we consider the model
$$
\lambda(t|Z^H(t)) = \lambda_0(t) \exp(\beta^T Z(t))
$$
then the partial likelihood function of $\beta$ for this model is given by
$$
PL(\beta) = \prod_u \left[ \frac{\exp(\beta^T Z_{I(u)}(u))}{\sum_{l=1}^n \exp(\beta^T Z_l(u))Y_(u)}\right]^{dN(u)}
$$
where $I(u)$ is the indicator variable that identifies the individual label for the 
individuals who fail at time $u$.




\subsection{Write a report on survival analysis}

\begin{itemize}
	\item Describe the event of interest (e.g., failure time)
	\item The start time and end time for the follow-up
	\item Type of censoring and the possible reasons for censoring
	\item The method for computing survival rate (Kaplan-Meier, or life table)
	\begin{itemize}
		\item median survival time, or 5-year surval rate (estimate and confidence interval)
		\item The statistical methods for comparing the survival rates (Logrank or Breslow) and 
			also the p-value.
	\end{itemize}
	\item Cox proportional hazards model for the relation between explantory variables and hazard:
	\begin{itemize}
		\item hazard ratios and the corresponding confidence interval
		\item hypothesis testing of the assumption of proportional hazards
	\end{itemize} 
\end{itemize}

\subsection{Competing Risk Model}

\section{Frailty model}

\section*{Exercises}

\begin{questions}

\question[40]
A clinical trial is intended to test a new treatment for malignant melanoma vs. current standard care. The
main outcome is disease-free survival. Each patient has a time $t$ in days after start of therapy that 
represents either the time of recurrence or death (if \texttt{status = 1}) or the end of the study (if \texttt{status 
= 0}). Each patient has two covariates we can use: \texttt{Tx = 1} if they are on the new therapy or \texttt{Tx 
= 0} if they are on standard of care. There are four subtypes of malignant melanoma, which we will characterize as 
\texttt{Type = A, B, C, or D}. The effect of the new therapy may differ among the four subtypes.

\begin{parts}
	\part We can estimate the survival curve for patients on the new and standard therapy including all four
		subtypes using the \texttt{Kaplan-Meier product limit estimator}. Suppose at a given time $t$ (in days after
		starting therapy), and for a particular subset of the patients, that there are 194 patients whose survival
		or censoring time is at least $t$. Suppose that 3 patients died or relapsed on day $t$ and 2 patients
		had a censoring time of $t$. By what fraction does the estimated survival curve drop at time $t$? How
		many patients are in the risk set just before $t$ and just after $t$?

	\part Write down the Cox model for predicting survival from \texttt{Tx, Type, and the Tx by Type interaction} 
		including a definition of the coefficients and their relationship to survival. State the important
		assumptions.

	\part If there are 20 patients at risk at a given time and 1 of them fails, write down the contribution (factor)
		to the partial likelihood from that failure time in terms of the model specification. Does it depend on
		the base hazard?

	\part Which will generate more accurate coefficient estimates, a study with 1000 patients of whom 900 survive 
		to the end of the study without recurrence, or a study with 500 patients 300 of whom survive? Why?

	\part Describe the most appropriate hypothesis test for whether the interaction term is required in the
		model. How the test statistic be calculated? To what specific statistical distribution would the test
		statistic be compared?

	\part Suppose that the reference levels of the covariates are \texttt{Tx = 0} and \texttt{Type = A}. List all 
		the coefficients in the model (symbolically). In terms of those coefficients, what would be the estimated 
		log hazard ratio of a patient with \texttt{Type B} melanoma on \texttt{Tx = 1} to a patient with \texttt{Type 
		C} melanoma on \texttt{Tx = 0}? What would be the estimated hazard ratio?

	\part How would you examine the proportionality assumption, graphically and/or with a statistical test?

	\part If it appears that the different subtypes have non-proportional hazards, how would you change the model 
		so that this could be accommodated?\\
	\textcolor{red}{Solution}\\
	One way would be to use a strata term for a grouping variable that was non-
		proportional. The alternative way is to include the time-dependent covariate.
\end{parts}


\question[55]
Consider the data from Prevention of Events with Angiotensin Converting Enzyme Inhibition
(PEACE) Trial. The goal of the study was to test whether ACE-inhibitor therapy,
when added to modern conventional therapy, would reduce the rate of nonfatal myocardial
infarction, death from cardiovascular causes, or revascularization in low-risk patients with
stable coronary artery disease and normal or slightly reduced left ventricular function. Patients
underwent randomization from November 1996 to June 2000 and were followed up for
as long as 7 years (median, 4.8 years), until December 31, 2003. The study was conducted
after approval from the institutional review boards at 187 sites in the United States (including
Puerto Rico), Canada, and Italy. Patients gave their written informed consent to
participate. An independent data and safety monitoring board reviewed patient safety data
and interim results. A morbidity and mortality review committee reviewed and classified all
outcomes. The data consist of the following variables
\begin{itemize}
	\item \texttt{t2death}: time to death (months)
	\item \texttt{death}: censoring status (1=death; 0=censored)
	\item \texttt{tx}: 0=standard 1=treatment
	\item \texttt{age}: age(years) at baseline
	\item \texttt{sysbp}: systolic blood pressure at baseline
	\item \texttt{gender}: 1=female; 0=male
	\item \texttt{hidiabet}: history of diabetes (1=yes; 0=no) at baseline
	\item \texttt{hihypert}: history of hypertension (1=yes; 0=no) at baseline
\end{itemize}

Load the dataset into R using
\begin{lstlisting}
peace=read.csv("data/peacedata.csv", head=T)
\end{lstlisting}

\begin{parts}

\part[5]
Conduct the logrank test to test the treatment effect of ACE-inhibitor therapy in
reducing mortality.

\part[10]
Estimate the hazard ratio of the ACE-inhibitor versus the standard care only and
construct the associated confidence interval based on the Cox regression model. Report
your findings. Compare the p-value of the treatment effect with that from the logrank
test. Why are they almost identical?

\part[10]
It is known that age, systolic blood pressure, gender, history of diabetes and history of
hypertension are associated with the survival time. Estimate the hazard ratio of the
the ACE-inhibitor versus the standard care only but adjusting for the aforementioned
factors, using the multivariate Cox regression model. Report your findings.

\part[10]
Estimate the hazard ratio of the ACE-inhibitor versus the standard care only and
construct the associated confidence interval based on the Cox regression model in male
and female patients, separately. Test if these two hazard ratios are identical. Report
and interpret your findings.

\part[20]
The clinical investigator decides to develop prognostic regression models using the
baseline age, systolic blood pressure, gender, history of diabetes and history of hypertension
to predict the survival time for patients receiving the conventional therapy
only and for patients receiving the ACE-inhibitor plus the conventional therapy. To
this end, one may build two separate Cox regression models in patient receiving the
conventional therapy only (tx=0) and in patient receiving the ACE-inhibitor plus the
conventional therapy (\texttt{tx=1}).

\begin{enumerate}[(a)]
	\item Plot the estimated survival functions for following four patients:
	\begin{itemize}
		\item patient A receiving the conventional therapy only 
			(age=60, sysbp=140, gender=1, hidiabet=0, hihypert=1)
		\item patient B receiving the ACE-inhibitor plus conventional therapy 
			(age=140, sysbp=60, gender=1, hidiabet=0, hihypert=1)
		\item patient C receiving the conventional therapy only 
			(age=60, sysbp=140, gender=0, hidiabet=0, hihypert=1)
		\item patient D receiving the ACE-inhibitor plus conventional therapy 
			(age=140, sysbp=60, gender=0, hidiabet=0, hihypert=1)
	\end{itemize}
	Would you give different treatment recommendations for a 60-year old male 
		patient, who has a systolic blood pressure of 140 and history of 
		hypertension but has no diabetes, and a female patient with the same 
		characteristics? Why?
	\item The researcher decides to use the restricted mean survival time (up to 
		80 months) to summarize the survival curve. What are the RMST for patients 
		A and B based on your estimated survival curves.
	\item You may use the resampling method to construct the 95\% confidence interval 
		for these two RMSTs. The basic idea is to replace $d M_i(t)$ by $d N_i(t)G_i$, 
		where $G_i \sim N(0, 1)$ generated by the users. Describe your procedure and 
		construct the corresponding 95\% confidence intervals.
\end{enumerate}

\end{parts}


\question[30]
(\textbf{Survival Analysis: Model Checking}) A subset of the Mayo PBC data is on the web as 
\texttt{mayo\_sub.dat}. This contains the 5 variables that were used in the final Mayo model as 
well as the survival time and status (also included is stage, but this doesn't appear to add 
to the prognostic potential – you can check this if you wish). Our question is: Do these 
data appear to satisfy the PH assumption? Note that formal methods (tests) were not fully 
developed until 1994, and the reported analysis using the Cox model appeared in 1989.

\begin{parts}
\part Fit the Mayo model and then assess whether the variables appear to satisfy the PH
assumption. Specifically, test the PH assumption for each variable. Interpret these tests.

\part For the variable (or variables) that are suggested to poorly satisfy the PH assumption
divide them into 3 groups and plot $log(− log(\hat{S}(t)))$ versus time. Interpret what this
plot suggests about whether the PH assumption is satisfied for the variable. Turn this plot in.

\part Fit the Cox model with the 5 Mayo model variables and plot the Schoenfeld residual
versus time for each variable. Use the smooth curve to help visualize trends. Interpret these
plots with respect to whether the PH assumption appears to be violated. Turn these plots in.

\part PBC legend has it that there is an observation which is an entry error (ie. the
value is wrong!), and that it has a large influence on one coefficient estimate. Create deltabeta’s
and plot these influence statistics against either time and/or the predictor variable
that they correspond to. Interpret these plots. Can you identify the error? (Note: see the
web page to obtain code to calculate the delta-beta’s).
\end{parts}

\question[20]

The file \texttt{addicts.dat} contains data regarding the time that heroin addicts remain in 
methadone treatment. In the lecture notes we found that the variable clinic did not satisfy 
the PH assumption and we were able to make inference on other predictor variables by using 
clinic as a stratifying variable.

These data were analyzed by Caplehorn and Bell (1991) who were interested in factors
associated with retaining subjects: "As methadone maintenance is of proven benefit only
to those in treatment, retention in treatment is an important measure of the effectiveness
of treatment programmes." and "To elucidate the reasons that programmes fail to retain
patients, we have studied the relationship between the maximum daily dose and retention
in a cohort of addicts." Scientific interest is in whether factors other than dose can be 
used to identify subjects at high risk for failing to be retained.

\begin{parts}
\part Calculate bivariate summaries for each of the predictor variables and their association
with time retained in treatment. Summarize these by creating a table of hazard ratios and
95\% CI’s for each variable when it is the single predictor in a Cox regression that uses clinic
as a stratifying variable.

\part Calculate a Cox regression model using all of the predictors. Summarize the results
by creating a single table of regression parameters (or hazard ratios) – use the computer
output as directly as possible in order to create this table. (again stratify on clinic)

\part Describe the assumptions in your Cox regression model, in particular what it means
to use clinic as a stratifying variable.

\part Are there other variables besides dose that appear to be predictive of retention failure?
Summarize the results of your analysis.

\part Do the other covariates appear to satisfy the PH assumption? Justify your conclusion

\end{parts}

\end{questions}

\end{CJK*}
\end{document}
