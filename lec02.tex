\documentclass[table,10pt]{beamer}

\mode<presentation>{
%\usetheme{Goettingen}
\usetheme{Boadilla}
\usecolortheme{default}
}
\usepackage{CJK}
\usepackage{graphicx}
\usepackage{amsmath, amsopn}
\usepackage{xcolor}
\usepackage[english]{babel}
\usepackage[T1]{fontenc}
\usepackage[latin1]{inputenc}
\usepackage{enumerate}
\usepackage{multirow}
\usepackage{url}
\ifx\hypersetup\undefined
	\AtbBeginDocument{%
		\hypersetup{unicode=true,pdfusetitle,
bookmarks=true,bookmarksnumbered=false,bookmarksopen=false,
breaklinks=false,pdfborder={0 0 0},pdfborderstyle={},backref=false,colorlinks=false}
	}
\else
	\hypersetup{unicode=true,pdfusetitle,
bookmarks=true,bookmarksnumbered=false,bookmarksopen=false,
breaklinks=false,pdfborder={0 0 0},pdfborderstyle={},backref=false,colorlinks=false}
\fi
\usepackage{breakurl}
\usepackage{color}
\usepackage{times}
\usepackage{xcolor}
\usepackage{listings}
\lstset{
language=R,
keywordstyle=\color{blue!70}\bfseries,
basicstyle=\ttfamily,
commentstyle=\ttfamily,
showspaces=false,
showtabs=false,
frame=shadowbox,
rulesepcolor=\color{red!20!green!20!blue!20},
breaklines=true}



\setlength{\parskip}{.5em}

\title[BI476]{BI476: Biostatistics - Case Studies}
\subtitle[observation]{Lec02: Observational Studies}
\author[Maoying Wu]{Maoying,Wu ({\small ricket.woo@gmail.com})}
\institute[CBB] % (optional, but mostly needed)
{
  \inst{}
  Dept. of Bioinformatics \& Biostatistics\\
  Shanghai Jiao Tong University
}
\date{Spring, 2018}

\AtBeginSection[]
{
  \begin{frame}<beamer>{Next Section ...}
    \tableofcontents[currentsection]
  \end{frame}
}

\begin{document}
\begin{CJK*}{UTF8}{gbsn}

\frame{\titlepage}

\begin{frame}
\frametitle{Goals of the lecture}
\begin{itemize}[<+->]
	\item To understand the cross-sectional study along its strength and weakness.
	\item To distinguish from the concepts of prevalence and incidence.
	\item To define the organization of a case-control study.
	\item To calculate and interpret the odds ratio (OR) as well as the confidence interval.
	\item To understand the strength and weakness of CCS.
	\item To understand the concept of bias and confounding.
	\item To define and organize a cohort study.
	\item To calculate and interpret the relative risk or risk ratio (RR).
	\item To understand the different biases, and know how to detect them.
	\item To understand the difference between association and causation.
\end{itemize}
\end{frame}

\begin{frame}
\frametitle{Outline}
	\tableofcontents
\end{frame}


\section{Some important concepts and notions}


\begin{frame}[t]
\frametitle{Measures of Disease Frequency}
A fundamental task in medical studies is to quantify or measure the \alert{occurrence of 
a disease in a population}. Obtaining a measure of the disease occurrence is one of the 
first step in understand the disease of interest.
\begin{description}[<+->]
	\item[\alert{Ratios}]{$\frac{a}{b}$, where $a$ is NOT part of $b$}
	\item[\alert{Proportions}]{$\frac{a}{b}$, where $a$ is INCLUDED in $b$}
	\item[\alert{Rates}]{$\frac{a}{b}$, where $a$ is the number of affected in a given time 
		interval, while $b$ is the population at risk over the same interval.}
\end{description}
\end{frame}

\begin{frame}[t]
\frametitle{Prevalence}
\textbf{Proportion of cases in the population at a given time}, indicating \alert{how widespread the disease is.}
$$
\textrm{Prevalence} = \frac{\mbox{\#cases observed at time } t}{\mbox{\#individuals at time} t}
$$
\begin{itemize}
	\item An outbreak of diarrhea on a cruise ship in a day.
	\item 8 persons out of 86 on the ship are exhibiting signs of diarrhea.
	\item The prevalence of diarrhea at this particular time is $8/86=0.092$
\end{itemize}
\end{frame}

\begin{frame}[t]
\frametitle{Incidence}
\begin{itemize}
	\item Cumulative incidence rate (CIR)
	$$
\textrm{CIR} = \frac{\mbox{\#newly disease cases in a time interval}}{\mbox{\#individuals-at-risk}}
	$$
	\begin{itemize}
		\item In the cruise ship example
		\item 12 persons developed diarrhoeal disease after 5 days of the outbreak.
		\item The indicidence rate was therefore $12/86=0.14$
	\end{itemize}
	\item Incidence density rate (IDR)
	\begin{itemize}
		\item \textbf{rate of occurrence of new cases}
		\item conveys information about the \alert{risk of 
			contracting the disease}. 
	\end{itemize}
	$$
\textrm{IDR} = \frac{\mbox{\#individuals of newly disease}}{\mbox{total time for all disease-free individuals-at-risk}}
	$$
\end{itemize}
\end{frame}

\begin{frame}
\frametitle{Outcome and Exposure}
A good research statement of a medical problem is often simple, considering one 
exposure and one outcome variable (e.g. organic solvents and brain tumor).

\begin{itemize}
	\item Primary outcome
	\item Secondary outcome
\end{itemize}

Research questions ask about the associations between exposures and outcomes. 
\begin{quote}
Is proximity of the residents to the coke-works site associated with hospital 
admission of children for repiratory problems?
\end{quote}

A hypothesis is then formulated to predict the result.
\begin{quote}
The risk of hospital admission for respiratory problems for children increases 
with proximity to the coke-works site.
\end{quote} 
\end{frame}


\begin{frame}[t]
\frametitle{Epidemiological Studies: Categories}
\begin{itemize}
	\item Descriptive Studies
	\begin{itemize}
		\item Correlational (Ecological) studies
		\item Case reports; Case series
		\item \alert{Cross-sectional studies}
	\end{itemize}
	\item Analytic Studies
	\begin{itemize}
		\item Case-cohort study
		\item Cohort study
		\item Intervention study
	\end{itemize}
	\item Time-dependent
	\begin{itemize}
		\item Retrospective study
		\item Prospective study
	\end{itemize}
\end{itemize}
\end{frame}


\begin{frame}[t]
\frametitle{Ecological Fallacy}
\framesubtitle{Higher fat intake causes breast cancer?}
\begin{figure}
\includegraphics[width=0.50\textwidth]{../images/ecological_fallacies.png}
\end{figure}
\begin{itemize}
	\item<1-> Ecological fallacy is an error in reasoning, usually based on 
		mistaken assumptions. 
	\item<2-> The ecological fallacy occurs when you \alert{make conclusions about 
		individuals based only on analyses of group data}. 
	\item<3-> In this fat intake example, we do not know individuals having 
		breast cancer were actually consumed high amount of fat.
\end{itemize}
\end{frame}

\section{Case Reports}

\begin{frame}[t]
\frametitle{Case Report: Example}
{\large \alert{Haemorrhagic-fever-like changes and normal chest radiograph in a doctor with SARS.}\\
\small{Wu E., et al. Lancet 2013; 361(9368): 1520-1521.}}

This case report is written by the patient, a physician who contracted SARS, 
and his colleague who treated him, during the 2003 outbreak of SARS in Hong Kong. 
They describe how the disease progressed in Dr. Wu and based on Dr. Wu’s case, 
advised that a chest CT showed hidden pneumonic changes and facilitate a rapid 
diagnosis.

{\large \alert{Kaposi's sarcoma in homosexual men-a report of eight cases.}}\\
\small{Hymes KB. et al. Lancet 1981; 2(8247): 598-600.}

This case report was published by eight physicians in New York city who had 
unexpectedly seen eight male patients with Kaposi’s sarcoma (KS). Prior to this, 
KS was very rare in the U.S. and occurred primarily in the lower extremities of 
older patients. These cases were decades younger, had generalized KS, and a much 
lower rate of survival. This was before the discovery of HIV or the use of the 
term AIDS and this case report was one of the first published items about AIDS 
patients.
\end{frame}


\begin{frame}[t]
\frametitle{Case Report: Definition}
Case report is an article that describes and interprets an individual case, often 
written in the form of a detailed story.

Case reports often describe:
\begin{itemize}
	\item Unique cases that cannot be explained by known diseases or syndromes
	\item Cases that show an important variation of a disease or condition
	\item Cases that show unexpected events that may yield new or useful information
	\item Cases in which one patient has two or more unexpected diseases or disorders
	\item the \alert{lowest-level} of evidence, but also the \alert{first-line} evidence.
\end{itemize}

A good case report will be clear about the importance of the observation being reported.

If multiple case reports show something similar, the next step might be a case-control 
study to determine if there is a relationship between the relevant variables.
\end{frame}


\begin{frame}[t]
\frametitle{Case Report: Pros and Cons}
\uncover<1->{\begin{alertblock}{\center Advantages}
\begin{itemize}
	\item Help the identification of new trends or diseases;
	\item Help detect new drug side effects and potential uses (adverse or beneficial)
	\item Educational - a way of sharing lessons learned
	\item Identifies rare manifestations of a disease
\end{itemize}
\end{alertblock}}
\uncover<2->{\begin{alertblock}{\center Disadvantages}
\begin{itemize}
	\item Cases may not be generalizable
	\item Not based on systmatic studies
	\item Causes or associations may have other explanations (confounding)
	\item Can be seen as emphasizing the bizarre or focusing on misleading elements.
\end{itemize}
\end{alertblock}}
\end{frame}

\begin{frame}[t]
\frametitle{Case Series: Example}
\framesubtitle{TNF inhibitors treatment vs. chronic cutaneous sarcoidosis}
\begin{figure}
\includegraphics[width=0.85\textwidth]{../images/case_series_example.png}
\end{figure}
\end{frame}

\begin{frame}[t]
\frametitle{Case Series: Example}
\framesubtitle{Bullous pemphigoid}
{\large \alert{Clinical and Immunological Profiles of 14 Patients With Bullous
Pemphigoid Without IgG Autoantibodies to the BP180 NC16A Domain}}\\
\small{Kenta N. et al. JAMA Dermatol. 2018;154(3):347-350.}

This case series examines the association of nonreactivity of IgG to the 
noncollagenous 16A domain using the enzyme-linked immunosorbent assay and 
the chemiluminescent enzyme immunoassay and severity of disease course in 
14 patients with bullous pemphigoid.

\begin{figure}
\includegraphics[width=.90\textwidth]{../images/bullous_pemphigoid.png}
\end{figure}
\end{frame}

\section{Cross-sectional study}

\begin{frame}[t]
\frametitle{Cross-sectional Studies: Definition}
\begin{itemize}
	\item A cross-sectional study explores the disease and risk factor 
		patterns in a representive part of the population, in a 
		narrowly defined time period.
	\item Primarily, this study provides information on \alert{prevalence} 
		of disease and risk factors.
	\item It also can seek associations, generate and test hypotheses. And, 
		by repetition, cross-sectional study can be used to measure 
		changes.
	\item Ideal cross-sectional study is of a geographically-defined, 
		representative sample of the population within a slice of 
		time and space.    
\end{itemize}
\end{frame}

\begin{frame}[t]
\frametitle{Cross-Sectional Studies: Example}
\framesubtitle{Association between sleep and hypertension}
{\large \alert{Insomnia with objective short sleep duration is associated 
with a high risk for hypertension.}}\\
\small{Vgontzas AN et al. Sleep 2009. 32:491-497.}
\begin{table}
\renewcommand\arraystretch{1.2}
\caption{Cross-sectional results of insomnia and hypertension}
\begin{tabular}{lccc}
\hline
 & Hypertension(+) & Hypertension(-) & \\
\hline
Insomnia(+) & 121 & 78 & 199\\
Insomnia(-) & 837 & 705 & 1542\\
\hline
 & 958 & 783 & 1741
\end{tabular}
\end{table}
Prevalence of hypertension: $958/1741 = 55\%$\\
Prevalence of insomnia: $199/1741 = 11\%$\\
Prevalence of hypertension among insomnia: $121/199 = 61\%$\\
Prevalence of hypertension among non-insomnia: $837/1542 = 54\%$\\
Prevalence ratio: $0.61/0.54 = 1.13$\\
Odds ratio: $\frac{121/78}{837/705} = $
\end{frame}

\begin{frame}[t]
\frametitle{Cross-Sectional Studies: Example}
\framesubtitle{The effect of sunless tanning products on tanning behaviors}
{\large \alert{A Cross-sectional Study Examining the Correlation Between 
Sunless Tanning Product Use and Tanning Beliefs and Behaviors.}}\\
\small{Rachel E. et al. Arch Dermatol. 2012;148(4):448-454.}
\begin{figure}
\includegraphics[width=0.90\textwidth]{../images/sunless_tanning.jpg}
\end{figure}
\begin{itemize}
	\item Time: May 30, 2007 - Dec 4, 2007
	\item Location: The Emory University campus and surroundings in Atlanta, 
		Georgia
	\item Participants: 415 community and university-affiliated women.
\end{itemize}
\end{frame}


\section{Case-control study}


\begin{frame}[t]
\frametitle{Case-Control Studies: Definition}
An epidemiological study in which a group of persons with the disease 
of interest (\alert{case}) and a group of persons with similar features 
to the case group but without the disease (\alert{control}) are selected 
to compare the proportion of persons exposed to a risk factor of interest 
(\alert{exposure}) in order to elucidate the causal relationship of the 
risk factor of interest and the disease.

From \textbf{case series}, \textbf{personal experience}, and \textbf{others stuffs},
\begin{itemize}
	\item Almost 99\% of patients suffered from condition $Y$ had a 
		history/evidence of exposure to $X$.
	\item What is the problem in this case study?
	\begin{itemize}
		\item It will never prove a causal relationship due to lack of 
			control group.
	\end{itemize}
	\item It will generate a hypothesis testable using epidemiological methods:
	\begin{itemize}
		\item Persons with disease $Y$ were more likely to have been exposed 
			to factor $X$ comparing to persons without the disease, but 
			similar in other aspects.
	\end{itemize} 
\end{itemize}
That's the case-control study, a retrospective, observational study.
\end{frame}

\begin{frame}[t]
\frametitle{Case-control studies}
\uncover<1->{\begin{table}
\footnotesize
\caption{$2\times 2$ contingency table}
\begin{tabular}{lcc}
\hline
 & Cases & Controls\\
\hline
Exposed(+) & $a$ & $b$\\
Exposed(-) & $c$ & $d$\\
\hline
\end{tabular}
\end{table}}

\uncover<2->{\begin{alertblock}{\center \small What we can answer}
\begin{itemize}
	\item What are the odds that a case was exposed? \alert{$a/c$}
	\item What are the odds that a control was exposed? \alert{$b/d$}
	\item What is the odds ratio? \alert{$\frac{a/c}{b/d} = \frac{ad}{bc}$}
\end{itemize}
\alert{Note}: OR is calculated in different fashion in pair-match case-control 
study. 
\end{alertblock}}
\uncover<3->{\begin{alertblock}{\center \small What we CAN'T answer}
\begin{itemize}
	\item the prevalence of disease in exposed and not-exposed.
	\item the incidences of disease in exposed and not-exposed.
	\item the relative risk to determine if there is an association 
		between the exposure and the disease.
\end{itemize}
\end{alertblock}}
\end{frame}

\begin{frame}[t]
\frametitle{Case-Control Studies: Computing Odds Ratio (OR)}
\begin{table}
\footnotesize
\caption{$2\times 2$ contingency table}
\begin{tabular}{lcc}
\hline
 & Cases & Controls\\
\hline
Exposed(+) & $a$ & $b$\\
Exposed(-) & $c$ & $d$\\
\hline
\end{tabular}
\end{table}

\uncover<2->{\begin{alertblock}{\center Computing OR}
\begin{itemize}
	\item The OR: $\widehat{\textrm{OR}} = \frac{a/c}{b/d} = \frac{ad}{bc}$
	\item The standard error: $\mathrm{SE}\log \textrm{OR} = \sqrt{\frac{1}{a} + \frac{1}{b} + \frac{1}{c} + \frac{1}{d}}$
	\item The 95\% confidence interval of $\log \textrm{OR}$: $\widehat{\log \textrm{OR}} \pm Z_{1-\alpha/2}^{-1} \mathrm{SE}\log \textrm{OR}$
	\item $p$-value can be computed using either \texttt{Chi-square test} or \texttt{Fisher's exact test}.
	\item In matched case-control study, \text{McNemar's test} is used to compute the $p$-value. 
\end{itemize}
\end{alertblock}}
\end{frame}


\begin{frame}[t]
\frametitle{Case-Control Study Issues}
\framesubtitle{Bias and confounding}
In case-control studies, the inferred association can be introduced by the 
subject selection process, the problem of data collection, and/or the other 
factor(s) that are related to both exposure and the outcome under study.

Only when the two concerns - \alert{bias and confounding} are addressed, the 
causality relationship can be derived.
\end{frame}


\begin{frame}[t]
\frametitle{Bias}
\begin{description}
	\item[Bias]{Any systematic error in the \underline{design/subject recruitment}, 
		\underline{data collection}, and/or \underline{data analysis} that 
		results in a mistaken estimation of the true exposure-outcome association.}
	\item[Selection Bias]{Error due to systematic differences in characteristics 
		between those selected/non-selected into a study; or systematic differences 
		in which cases and controls, exposed and non-exposed subjects are selected 
		so that distorted association is observed.}
\end{description}
\uncover<1->{\begin{exampleblock}{Smoking-AMI Association}
\begin{itemize}
	\item If hospitalized cases of AMI are selected as cases, and heavy/long-term smokers are more 
likely to die outside of hospital (massive MI), the smoking-MI association would be 
underestimated.

	\item If hospitalized AMI and hostpitalized non-MI patients are selected as cases and controls, 
and smokers are more likely to be hospitalized for other condition (e.g. pulmonary), the 
smoking-AMI association would be underestimated.
\end{itemize}
\end{exampleblock}}
\end{frame}

\begin{frame}[t]
\frametitle{Selection Bias}
\begin{itemize}
	\item Exposure has influence on the process of case assessment: the 
		exposure prevalence in cases is biased.
	\item Exposure has influence on the process of control selection: the 
		exposure prevalence in controls is biased (e.g. use chronic 
		bronchitis patients as controls for study of smoking and CHD, 
		smoking and LC.)
	\item \alert{Selective survival} and \alert{selective migration}: the 
		exposure prevalence in prevalent cases may be biased compared 
		to incident cases. 
\end{itemize}
\uncover<2->{\begin{alertblock}{Case-control association is valid only when}
\begin{itemize}
	\item Cases are selected as \alert{representative sample} of all people 
		with such disease.
	\item Controls are selected as a representive sample of all people 
		without the disease in the same population as cases. 
\end{itemize}
\end{alertblock}}
\end{frame}

\begin{frame}[t]
\frametitle{Information Bias}
\framesubtitle{Misclassification bias}
\begin{description}
	\item[Information bias]{Systematic error due to inaccurate measurement and/or 
		classification of study variables. It can occur in the classification 
		of outcome, exposures, and covariates.}
\end{description}
\uncover<2->{\begin{exampleblock}{Example}
\begin{description}
	\item[recall bias]{Lung cancer cases are more likely to recall their smoking 
		history (quantitatively or qualitatively), which will lead to over-estimation 
		of smoking-cancer relationship.}
	\item[missing bias]{Overweighed persons are more likely to have missing data in 
		untrasound examination of carotid arteries.}
	\item[report bias]{Females are more likely to under-report of their weight, and males 
		over-report.}
\end{description}
\end{exampleblock}}
\end{frame}

\begin{frame}[t]
\frametitle{Confounders (Confounding)}
\alert{Confounding} is a situation in which a measure of the association between 
exposure and outcome is distorted because of the relationship between the exposure 
and the other factor (the confounder) that influences the outcome under study.

Confounding is the most important concept because it impacts the \textbf{validity} 
of an observational study.

\uncover<2->{\begin{alertblock}{Factor $X$ is a confounder only if it meets the 3 criteria}
\begin{itemize}
	\item It is an risk factor of the outcome under study, independent of the risk 
		factor under study.
	\item It is associated with the exposure under study.
	\item It is not in the causal pathway between the exposure and the outcome. 
\end{itemize}
\end{alertblock}}
\end{frame}


\begin{frame}[t]
\frametitle{How to avoid confounding association?}
\uncover<1->{\begin{alertblock}{In the design and conduct of study}
\begin{itemize}
	\item \alert{Matching} (individual or group) by potential confounders - in 
		case-control studies.
	\item Collect \alert{information} on potential confounders. 
\end{itemize}
\end{alertblock}}
\uncover<2->{\begin{alertblock}{In the analysis of study}
\begin{itemize}
	\item \alert{Conceptualize} your potential confounders.
	\item \alert{Stratify} by potential confounders to identify major confounders.
	\item Stratify to derive stratum-specific exposure-outcome association.
	\item Statistical \alert{adjustment} of confounders.
\end{itemize}
\end{alertblock}}
\end{frame}


\begin{frame}[t]
\frametitle{Matched Case-Control Studies}
\begin{itemize}
	\item<1-> Select controls who are identical (similar) to cases on potential confounders.
	\begin{itemize}
		\item<2-> \alert{Pair/Individual} match or \alert{frequency} match
		\item<3-> Better control for confounding, especially when the distribution of a 
			confounder do not have much overlap between the cases and source 
			population (e.g., if cases of myocardial infaction tends to be older)
		\item<4-> Analytical methods: \alert{Matched paired analysis} or \alert{conditional 
			logistic regression}
	\end{itemize} 
\end{itemize}
\uncover<5->{
\begin{table}
\caption{Pair-matched data}
\begin{tabular}{lcc}
\hline
\multirow{2}{*}{Cases} & \multicolumn{2}{c}{Controls} \\
\cline{2-3}
& Exposed & Non-exposed\\
\hline
Exposed & $a$ & \alert{$b$}\\
Non-exposed & \alert{$c$} & $d$\\
\hline
\end{tabular}
\end{table}}
\uncover<5->{\alert{How to compute the odds ratio (OR)?}}
\uncover<6->{$$
\textrm{OR} = \frac{b}{c}
$$}
\end{frame}


\begin{frame}[t]
\frametitle{Matched Paired Data Analsysis}
\begin{table}
\scriptsize
\caption{Pair-matched data}
\begin{tabular}{lcc}
\hline
\multirow{2}{*}{Cases} & \multicolumn{2}{c}{Controls} \\
\cline{2-3}
& Exposed & Non-exposed\\
\hline
Exposed & $a$ & \alert{$b$}\\
Non-exposed & \alert{$c$} & $d$\\
\hline
\end{tabular}
\end{table}
\uncover<2->{
\begin{alertblock}{\center \small Computing OR}
\begin{itemize}
\scriptsize
	\item $a$ and $d$ are \alert{concordant pairs}; $b$ and $c$ are \alert{discordant pairs}.
	\item The odds ratio: $\widehat{\mathrm{OR}} = \frac{b}{c}$
	\item The standard error: $\mathrm{SE}_{\log \hat{\mathrm{OR}}} = \sqrt{\frac{1}{b} + \frac{1}{c}}$
	\item The confidence interval of $\log \mathrm{OR}$:
	$$
\log \widehat{\mathrm{OR}} \pm Z_{1-\alpha/2}\mathrm{SE}_{\log \hat{\mathrm{OR}}}
	$$
	\item The \alert{regular McNemar's Chi-square test}:
	$$
\chi_{\textrm{McN}}^2 = \frac{(b-c)^2}{b+c} \sim \chi_{df=1}^2
	$$
	\item The \alert{continuity-correct McNemar's Chi-square test}:
	$$
\chi_{\textrm{McN}}^2 = \frac{(|b-c|-1)^2}{b+c} \sim \chi_{df=1}^2
	$$
\end{itemize}
\end{alertblock}}
\end{frame}

\begin{frame}[t]
\frametitle{A Paired Match Analysis}
\framesubtitle{estrogen and endometrial carcinoma}
{\large \alert{Association Of Exogenous Estrogen And Endometrial Carcinoma}}\\
\small{Donald C. Smith et al. NEJM 1975 Dec 4;293(23):1164-7.}
\begin{itemize}
	\item Retrospective case-control study;
	\item 317 patients with endometrium carcinoma
	\item 317 matched controls with other gynecologic neoplasms
\end{itemize}
\begin{table}
\renewcommand\arraystretch{1.2}
\small
\caption{Paired study of association between estrogen and endometrial carcinoma}
\begin{tabular}{lccc}
\hline
 & \multicolumn{2}{c}{Controls} & \\
\cline{2-3}
Cases & Exposed & Non-exposed & Total\\
\hline
Exposed & 39 & \alert{113} & 152\\
Non-exposed & \alert{15} & 150 & 165\\
\hline
Total & 54 & 263 & 317
\end{tabular}
\end{table}
\uncover<2->{
$$
\textrm{OR} = 113/15 = 7.5
$$
}
\end{frame}


\begin{frame}[t]
\frametitle{Estrogen and Endometrial Carcinoma}
\framesubtitle{Misuse of unmatched analysis}
{\large \alert{Association Of Exogenous Estrogen And Endometrial Carcinoma}}\\
\small{Donald C. Smith et al. NEJM 1975 Dec 4;293(23):1164-7.}
\begin{table}
\renewcommand\arraystretch{1.2}
\footnotesize
\caption{Unmatched study of association between estrogen and endometrial carcinoma}
\begin{tabular}{lccc}
\hline
 & Cases & Controls & Total\\
\hline
Exposed & 152 & 54 & 206\\
Non-exposed & 165 & 263 & 428\\
\hline
Total & 317 & 317 & 634
\end{tabular}
\end{table}
\uncover<2->{
$$
\textrm{OR} = \frac{152/54}{165/263} = 4.5
$$
}
\uncover<3->{\begin{alertblock}{\center \small Conclusion}
\begin{itemize}
	\item The unmatched $\textrm{OR}$ (4.5) is biased towards null compared 
		to the matched one (7.5).
	\item The stronger the confounders, the biased the association. 
\end{itemize}
\end{alertblock}}
\end{frame}

\begin{frame}[t]
\frametitle{Case-control Studies: Strengths and Limitations}
\uncover<1->{\begin{alertblock}{\center Advantages}
\begin{itemize}
	\item Small sample size (rare disease)
	\item Less time (disease with long induction and latent period)
	\item Less expensive (small N, efficient for exposure that is 
		expensive to measure)
\end{itemize}
\end{alertblock}}

\uncover<2->{\begin{alertblock}{\center Disadvantages}
\begin{itemize}
	\item Inefficient for rare exposure
	\item Selection bias (Are cases representative? Do controls 
		represent the source population?)
	\item Challenge in measurement of exposure (measure exposure 
		after the occurrence of disease)
	\item Difficulty in determine temporality
	\item Only one outcome is studied
	\item Incidence of disease cannot be studied, though OR is 
		intended to estimate incidence ratio
\end{itemize}
\end{alertblock}}
\end{frame}

\begin{frame}[t]
\frametitle{Exercise: Case-control study}
\framesubtitle{Tabacco use and acoustic neuroma}
{\large \alert{Role of tobacco use in the etiology of acoustic neuroma}}\\
\small{Palmisano S et al. Am J Epidemiol. 2012 Jun 15;175(12):1243-51.}
\begin{itemize}
	\item What kind of study? \alert{case-control study}
	\item How many cases and controls? \alert{451 cases} and \alert{710 
		population-based controls.}
	\item How did the author match the cases and controls?
	\item Can you write down the $2\times 2$ contingency tables here?
	\item How did the author analyze the data?
	\item What was the effect size used in this article? and the confidence 
		interval?
	\item Why did the author analyze the data in males and females, respectively?
\end{itemize}
\end{frame}

\section{Cohort Study}

\begin{frame}[t]
\frametitle{Cohort study}
\begin{itemize}
	\item Observational study, with selection into study on basis of exposure 
		status at the beginning of the study
	\item Either prospective study or retrospective study
\end{itemize}
\end{frame}


\begin{frame}
\frametitle{Flu Vaccine Study}
\framesubtitle{An epidemiology case study}
{\large \alert{Influenza Vaccination and Reduction in Hospitalizations
for Cardiac Disease and Stroke among the Elderly.}}\\
\small{Kristin Nichol et al.: NEJM 2003;348:1322-32.}

These investigators used the administrative data bases
of three large managed care organizations to study the
impact of vaccination in the elderly on hospitalization
and death. Administrative records were used to whether
subjects had received influenza vaccine and whether
they were hospitalized or died during the year of study.
\end{frame}


\begin{frame}[t]
\frametitle{Flu Vaccine Study}
\framesubtitle{Summarization table}
The table below summarizes findings during the \underline{1998-1999} flu season.
\begin{table}
\renewcommand\arraystretch{1.2}
\footnotesize
\caption{flu vaccine study data in 1998-1999}
\begin{tabular}{p{3cm}cc}
\hline
& Vaccinated & Unvaccinated\\
& $N=77,738$ & $N=62,217$ \\
\hline
Hospitalized due to pneumonia or influenza & 495 & 581\\
\hline
Hospitalized due to cardiac disease & 888 & 1026\\
\hline
Deaths & 943 & 1361\\
\hline
\end{tabular}
\end{table}
\alert{If the exposure is vacination and outcome of interest is death, how to assess the 
association?}
\end{frame}

\begin{frame}[t]
\frametitle{Flu Vaccine Study}
\framesubtitle{Contingency table}
\begin{table}
\renewcommand\arraystretch{1.2}
\rowcolors{2}{gray!25}{white}
\scriptsize
\caption{flu vaccine study data in 1998-1999}
\begin{tabular}{p{3cm}cc}
\hline
& Vaccinated & Unvaccinated\\
& $N=77,738$ & $N=62,217$ \\
\hline
Hospitalized due to pneumonia or influenza & 495 & 581\\
\hline
Hospitalized due to cardiac disease & 888 & 1026\\
\hline
Deaths & 943 & 1361\\
\hline
\end{tabular}
\end{table}
\alert{If the exposure is vacination and outcome of interest is death, how to assess the 
association?}
\begin{table}
\rowcolors{2}{white}{gray!25}
\renewcommand\arraystretch{1.2}
\small
\caption{flu vaccine and deaths in 1998-1999}
\begin{tabular}{lccc}
\hline
 & Dead & Alive & \\
\hline
Vaccinated & 943 & 76,795 & 77,738\\
Non-vaccinated & 1361 & 60,856 & 62,217\\
\hline
\end{tabular}
\end{table}
\end{frame}

\begin{frame}[t]
\frametitle{Solutions}
\framesubtitle{Relative Risk (RR) and Risk Difference (RD)}
\begin{table}
\rowcolors{2}{white}{gray!25}
\renewcommand\arraystretch{1.2}
\scriptsize
\caption{A general $2 \times 2$ contingency table}
\begin{tabular}{lccc}
\hline
 & Disease & Non-Disease & \\
\hline
Exposed & $a$ & $b$ & $r_1$\\
Non-exposed & $c$ & $d$ & $r_2$\\
\hline
\end{tabular}
\end{table}
\begin{columns}
\footnotesize
\begin{column}{0.45\textwidth}
\begin{block}{\center RR approach}
\textcolor{red}{$$
\textrm{RR} = \frac{a/r_1}{c/r_2}
$$}
\begin{itemize}
	\item measuring the strength of the association.
	\begin{itemize}
	\scriptsize
		\item $\textrm{RR}=1$ suggests no association.
		\item $\textrm{RR} \to 1$ suggests weak association.
		\item $|\textrm{RR}|>>1$ suggests a strong association.
	\end{itemize} 
\end{itemize}
\end{block}
\end{column}
\begin{column}{0.45\textwidth}
\begin{block}{\center RD approach}
\textcolor{red}{$$
\textrm{RD} = \frac{a}{r_1} - \frac{c}{r_2}
$$}
\begin{itemize}
	\item a better measure of \alert{public health impact}.
	\item How much impact would a prevention have?
	\item How many people would benefit?
\end{itemize}
\end{block}
\end{column}
\end{columns}
\end{frame}


\begin{frame}[t]
\frametitle{Risk Ratio (RR)}
\framesubtitle{measuring the association between vaccination and death}
\begin{table}
\rowcolors{2}{white}{gray!25}
\renewcommand\arraystretch{1.2}
\small
\caption{flu vaccine and deaths in 1998-1999}
\begin{tabular}{lccc}
\hline
 & Dead & Alive & \\
\hline
Vaccinated & 943 & 76,795 & 77,738\\
Non-vaccinated & 1361 & 60,856 & 62,217\\
\hline
\end{tabular}
\end{table}
$$\textrm{RR} = \frac{\textrm{CI}_e}{\textrm{CI}_u} = \frac{943/77738}{1361/62217} = 0.554$$
\begin{alertblock}{\center Conclusion}
\begin{itemize}
	\item There is a strong association between flu vaccination and death?
	\item Vaccination can protect ...?
	\item How about the hospitalization due to two diseases?
\end{itemize}
\end{alertblock}
\end{frame}


\begin{frame}[t]
\frametitle{Risk Ratio: Confidence Interval}
\begin{table}
\rowcolors{2}{white}{gray!25}
\renewcommand\arraystretch{1.2}
\scriptsize
\caption{A general $2 \times 2$ contingency table}
\begin{tabular}{lccc}
\hline
 & Disease & Non-Disease & \\
\hline
Exposed & $a$ & $b$ & $r_1$\\
Non-exposed & $c$ & $d$ & $r_2$\\
\hline
\end{tabular}
\end{table}
\uncover<2->{\begin{alertblock}{\center Confidence interval}
\begin{itemize}
	\item $\hat{p_1} = \frac{a}{r_1}$; $\hat{p_0} = \frac{c}{r_2}$
	\item $\widehat{\mathrm{RR}} = \frac{p_1}{p_0}$
	\item $\mathrm{SE}_{\log \mathrm{RR}} = \sqrt{\frac{1-\hat{p_1}}{r_1 \hat{p_1}} + 
		\frac{1-\hat{p_0}}{r_2 \hat{p_0}}}$
	\item The confidence interval of $\log \mathrm{RR}$: $\log \widehat{\mathrm{RR}} \pm 
		z_{1-\alpha/2} \mathrm{SE}_{\log \mathrm{RR}}$
	\item Hypothesis testing: \alert{Chi-square test} or \alert{Fisher's exact test}
\end{itemize}
\end{alertblock}}
\end{frame}


\begin{frame}[t]
\frametitle{Risk Difference (RD)}
\framesubtitle{Attributable risk}
\begin{table}
\rowcolors{2}{white}{gray!25}
\renewcommand\arraystretch{1.2}
\small
\caption{flu vaccine and deaths in 1998-1999}
\begin{tabular}{lccc}
\hline
 & Dead & Alive & \\
\hline
Vaccinated & 943 & 76,795 & 77,738\\
Non-vaccinated & 1361 & 60,856 & 62,217\\
\hline
\end{tabular}
\end{table}
$$\textrm{RD} = \textrm{CI}_e - \textrm{CI}_u = \frac{943}{77738} - \frac{1361}{62217} = -0.0097 = -97/100000$$ per year.
\begin{alertblock}{\center Conclusion}
\begin{itemize}
	\item Flu vacinnation can reduce the death rate by 97 per 100,000 population per year.
	\item Flu vaccination has a strong protective effect ....?  
	\item How about the hospitalization due to two diseases?
\end{itemize}
\end{alertblock}
\end{frame}

\begin{frame}[t]
\frametitle{Different conclusions for RR and RD}
\framesubtitle{Sometimes ...}
\begin{table}
\caption{Annual Mortality Per 100,000 (CI)}
\begin{tabular}{p{3cm}cc}
\hline
 & LC & CHD\\
\hline
Smokers & 140 & 669\\
Non-smokers & 10 & 413\\
\hline
RR & 14 & 1.6\\
RD & 130 & 256\\
\hline
\end{tabular}
\end{table}
\center\alert{Note: LC - lung cancer; CHD - coronary heart disease}
\begin{alertblock}{\center Conclusion}
\begin{itemize}
	\item Smoking is a stronger risk factor for ...?
	\item Smoking is a bigger public health problem for ...?
\end{itemize}
\end{alertblock}
\end{frame}


\begin{frame}[t]
\frametitle{Benefits and Risks for Different Diseases}
\framesubtitle{Protective and risk effects of aspirin}
\begin{table}
\renewcommand\arraystretch{1.2}
\caption{RRs and RDs of aspirin on some heart diseases}
\begin{tabular}{lcccc}
 & Aspirin & Placebo & Risk & Risk\\
 & (/10,000) & (/10,000) & Ratio & Diff\\
\hline
MI & 125.9 & 216.6 & 0.59 & -100\\
\hline
Stroke & 107.8 & 88.8 & 1.2 & 19\\
- Ischemic & 82.4 & 74.3 & 1.1 & 8\\
- Hemorrhagic & 20.8 & 10.9 & 1.9 & 10\\
\hline
Upper GI ulcer & 153.1 & 125.1 & 1.2 & 28\\
- with hemorrhage & 34.4 & 19.9 & 1.7 & 15\\
\hline
Bleeding & 2699.1 & 2037.3 & 1.3 & 690\\
- Transfusion need & 43.5 & 25.4 & 1.7 & 18\\
\hline
\end{tabular}
\end{table}
\center \alert{The risk difference is in /10,000 persons per year.}
\end{frame}

\begin{frame}[t]
\frametitle{Rare Outcomes - RR or RD?}
\framesubtitle{It depends ...}
If we are going to discuss rare, but severe possible complications of influenza vaccine, would 
it be better to look at the \alert{RR} or the \alert{RD}?
\begin{alertblock}{\center Observed frequencies}
\begin{itemize}
	\item Exposed people: 2/100,000
	\item Non-exposed people: 1/100,000
\end{itemize}
\end{alertblock}
\begin{alertblock}{\center Choices}
\begin{itemize}
	\item \alert{RR=2}: those exposed had two times the risk! (\alert{OMG!})
	\item \alert{RD=1/100,000}: the exposed group had an excess risk of 1 case 
		per 100,000 subjects (\alert{NO THAT IMPORTANT!}). 
\end{itemize}
\end{alertblock}
\end{frame}


\begin{frame}[t]
\frametitle{Attributable Risk \%}
\framesubtitle{Attributable proportion}
\alert{\large{What \% of risks in the exposed group
can be attributed to having had the exposure?}}

The proportion (\%) of disease in the exposed group
that can be attributed to the exposure, i.e., the
proportion of disease in the exposed group that
could be prevented by eliminating the exposure:

$$
\textrm{AR}\% = \frac{\textrm{RD}}{I_e} = \frac{0.053-0.013}{0.053}\times 100 = 75\% 
$$

\begin{alertblock}{\center Interpretation} 
\textcolor{blue}{\large 75\% of risks occurring in patients who had the exposure could be 
attributable to the exposure.}
\end{alertblock}
\end{frame}

\begin{frame}[t]
\frametitle{Cohort study: Pros and cons}
\begin{block}{Strengths}
\begin{itemize}
	\item Accurate exposure
	\item Subjects in cohorts can be matched, limiting the influence of confounders.
	\item Temporal relationship: short-term or long-term outcome
	\item Multiple outcomes: cerebral dysfunction, myocardial infarction, etc.
	\item Very rich study design, but it is diffcult and often act as the second step.
	\item Easier and chpeaper than an RCTs.
\end{itemize}
\end{block}
\begin{block}{Limitations}
\begin{itemize}
	\item Cohorts can be difficult to identify due to confounders.
	\item No randomization leads to imbalance in patient features.
	\item Blinding/masking is difficult.
	\item More difficulties related to both costs and feasibility because of the 
		longitudinal follow-ups
	\item Different lost-to-follow-up (LTFU) rates for healthier and diseased groups
\end{itemize}
\end{block}
\end{frame}


\begin{frame}[t]
\frametitle{Limitations of Observational Studies}
\begin{itemize}
	\item Patients who receive on-pump or off-pump CABG may differ in many ways 
		besides their type of surgery that could be related to the outcome under 
		study (confounding).
	\item In observational studies, on- vs. off-pump CABG was related to:
	\begin{itemize}
		\item extent and location of heart disease
		\item obesity
		\item chronic obstructive pulmonary disease (COPD)
		\item experience of practitioner with off-pump
		\item etc.
	\end{itemize}
	\item These differences could explain differences in rates of endpoints such 
		as mortality or complications seen in observational studies between on- vs. 
		off-pump patients.
	\item We can control for known confounders through univariate stratification, 
		or multivariate analysis (MVA), but we can't control for unknown, unmeasured, 
		or unmeasurable differences.
	\item The amount of uncontrolled bias and confounding could be as large as the 
		effect under assessment.
\end{itemize}
\end{frame}


\begin{frame}[t]
\frametitle{Propensity score}
\framesubtitle{A method to minimize the confounding in observational studies}
\begin{itemize}
	\item In RCTs, the proper randomization provides a \alert{balanced distribution} of both observed 
		and unobserved factors between different study arms, which \alert{minimize confounding}.
	\item Observational studies often suffer because such balance cannot be met.
	\item Multivariate regression can provide adjusted estimate for the main exposure by 
		including the confounders in a model.
	\item However, when more than the allowable number of covariates need to be adjusted, we 
		may need to consider dimensionality reduction.
	\item Propensity score adjustment is a method of control for potential confounders for the 
		association between a pre-specified exposure and outcome, which provides a \alert{
		quasi-RCT} setting for observational data analysis. 
\end{itemize}
The propensity score analysis works by creating a single score estimate on the probability to 
exposure, which is used to control for many potential confounders at once without a loss of 
analytical power, is considered the most aggressive adjustment for measured confounders.
\end{frame}


\begin{frame}[t]
\frametitle{Aspirin Use and All-Cause Mortality}
\framesubtitle{A propensity score matching and adjustment}
{\large \alert{Aspirin Use and All-Cause Mortality Among Patients Being Evaluated for Known or Suspected 
Coronary Artery Disease: A Propensity Analysis}}\\
\small{Gum PA et al. JAMA, 2001 Sep 12; 286(10):1187-94.}
\begin{itemize}
	\item Prospective cohort study
	\item Cleveland Clinic, 1990-1998
	\item Mean follow-up 3.1 years
	\item 6174 patients undergoing stress echocardiography for evaluation of known or suspected 
		coronary disease
\end{itemize}
\end{frame}


\begin{frame}[t]
\frametitle{Exercise: Propensity Score Matching}
\framesubtitle{Association of statin use and cataracts}
{\large \alert{Association of statin use with cataracts: a propensity score-matched analysis.}}\\
\small{Leuschen J, et al. JAMA Ophthalmol. 2013 Nov;131(11):1427-34.}
\begin{itemize}
	\item What was the primary outcome?
	\item What was the exposure in this study?
	\item Which study design did this study adopt?
	\item Why do we say it is a cohort study other than a case-control study?
	\item What were the variables used to compute the propensity scores? Why and how?
	\item How did the authors compare the baseline characteristics between statin-users 
		and non-statin users?
	\item Why did the authors conduct sensitivity analysis?
	\item What is the difference between primary analysis and secondary analysis?
	\item Read the comments and response supplementing this article, and give your 
		own comments.
\end{itemize}
\end{frame}

\begin{frame}[t]
\frametitle{Observational Studies: Summary}
\begin{itemize}
	\item<1-> Cross-sectional Study
	\item<2-> Case-control study
	\begin{itemize}
		\item<2-> Matched case-control study
	\end{itemize}
	\item<3-> Cohort study
\end{itemize}
\uncover<4->{\begin{alertblock}{Important issues}
\begin{itemize}
	\item<5-> Low response rate in cross-sectional study
	\item<6-> Low follow-up rate in cohort study
	\item<7-> Selection bias in case-control study
	\item<8-> Information (misclassification) bias in all studies
	\item<9-> Confounding in all studies
	\item<10-> Use the most appropriate analytical methods
\end{itemize}
\end{alertblock}}
\end{frame}


\begin{frame}[t]
\frametitle{Design a Study Protocol}
\begin{enumerate}[(A)]
	\item What is your Research Question and Hypothesis 
	\item Target population
	\item Explanatory variable (Exposure)
	\item Response variable (Outcome)
	\item Extraneous variable (Potential confounders)
	\item Anatomy of the design
	\item Planned analyses
	\item Sample size requirements
	\item Study limitations
\end{enumerate}
\end{frame}


\begin{frame}[t]
\frametitle{Statistical Testing Choices}
\uncover<1->{\begin{alertblock}{Continuous outcome (mean)}
\begin{itemize}
	\item Two independent groups
	\begin{itemize}
		\item $t$-test (parametric)
		\item Wilcoxon Mann-Whitney rank-sum test (nonparametric)
	\end{itemize}
	\item 2+ independent groups
	\begin{itemize}
		\item ANOVA (parametric)
		\item Kruskal-Wallis test (nonparametric)
	\end{itemize}
	\item Paired data
	\begin{itemize}
		\item Paired $t$-test (parametric)
		\item Wilcoxon signed-rank test (nonparametric)
	\end{itemize}
	\item 2+ dependent groups
	\begin{itemize}
		\item ANOVA
		\item Friedman's test
	\end{itemize}
\end{itemize}
\end{alertblock}}
\uncover<2->{\begin{alertblock}{Categorical outcome (Percentage)}
\begin{itemize}
	\item Independent groups (2+)
	\begin{itemize}
		\item $\chi^2$-test
		\item Fisher's exact test
	\end{itemize}
	\item Paired data ($2 \times 2$)
	\begin{itemize}
		\item McNemar's test
	\end{itemize}
\end{itemize}
\end{alertblock}}
\end{frame}

\begin{frame}
\center{\LARGE{Questions?}}
\end{frame}

\end{CJK*}
\end{document}
